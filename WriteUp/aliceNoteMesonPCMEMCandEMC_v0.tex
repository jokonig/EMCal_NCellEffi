
%\documentclass[11pt]{elsart}
\documentclass[ALICE]{ALICE_analysis_notes}

% Use the option doublespacing or reviewcopy to obtain double line spacing
%\documentclass[doublespacing]{elsart}
\usepackage[american]{babel}
\usepackage[utf8x]{inputenc}
\usepackage[colorlinks=true,bookmarksnumbered=true,bookmarksopen=true]{hyperref}

 \usepackage{graphicx}
 \usepackage{subfigure}
 \usepackage{geometry}
% %\geometry{a4paper,inner=28mm, outer=18mm, top=25mm, bottom=30mm,twoside}
 \geometry{a4paper, inner=28mm, outer=18mm, top=25mm, bottom=29mm, headsep=10mm, footskip=12mm, twoside}
 \usepackage{amsmath}
 \usepackage{amssymb}
 \usepackage{mathcomp}
 \usepackage{array}
\usepackage{helvet}
 \usepackage{color}
% 	\usepackage{hyperref}
\usepackage[dvips]{epsfig}
\hypersetup{
    pdfauthor={Daniel Mühlheim},     % author
    colorlinks=true,       % false: boxed links; true: colored links
    linkcolor=blue,        % color of internal links (change box color with linkbordercolor)
    citecolor=black,       % color of links to bibliography
    filecolor=magenta,     % color of file links
    urlcolor=blue          % color of external links
}
\usepackage[all]{hypcap}
 \usepackage{listings}
 \usepackage{cite}
 \usepackage[printonlyused]{acronym}
 \usepackage{units}
 \usepackage{lineno}
\linenumbers
%  \usepackage[numbers]{natbib}
 \usepackage{ascii}
 \usepackage{feynmf}
\usepackage{dcolumn}
\usepackage[point,rounding]{rccol}
\usepackage{rotating}

\renewcommand*{\acsfont}[1]{{\color{black}#1}}
\renewcommand*{\acffont}[1]{{\color{black}#1}}
 \usepackage{booktabs}
\setlength{\abovecaptionskip}{4pt plus 0pt minus 0pt}
\setlength{\parskip}{0pt}

\makeatletter
\AtBeginDocument{%
  \renewcommand*{\AC@hyperlink}[2]{%
    \begingroup
      \hypersetup{hidelinks}%
      \hyperlink{#1}{#2}%
    \endgroup
  }%
}
\makeatother
\PHnumber{ALICE-ANA-XXXX}


\renewcommand{\floatpagefraction}{.82}
\renewcommand{\bottomfraction}{.95}
\renewcommand{\topfraction}{.95}
\renewcommand{\textfraction}{.04}
\newcommand{\pT}{$p_{\mbox{\tiny T}}$\xspace}
\newcommand{\pTHard}{$p_{\mbox{\tiny T,hard}}$\xspace}
\newcommand{\sNN}{$\sqrt{s_{\mbox{\tiny NN}}}$\xspace}
\newcommand{\sNNF}{$\sqrt{s_{\mbox{\tiny NN}}}=2.76\,$TeV\xspace}
\newcommand{\sNNMa}{$\sqrt{s_{\mbox{\tiny NN}}}=5.02\,$TeV\xspace}
\newcommand{\s}{$\sqrt{s}$\xspace}
\newcommand{\sth}{\s~=~13~TeV\xspace}
\newcommand{\sei}{\s~=~8~TeV\xspace}
\newcommand{\stw}{\s~=~2.76~TeV\xspace}
\newcommand{\mT}{$m_{\mbox{\tiny T}}$\xspace}
\newcommand{\qT}{$q_{\mbox{\tiny T}}$\xspace}
\newcommand{\xT}{$x_{\mbox{\tiny T}}$\xspace}
\newcommand{\Pb}{{\mbox{Pb--Pb}}\xspace}
\newcommand{\AACol}{{\mbox{A--A}}\xspace}
\newcommand{\Au}{{\mbox{Au--Au}}\xspace}
\newcommand{\pPb}{{\mbox{p--Pb}}\xspace}
\newcommand{\pA}{{\mbox{p--A}}\xspace}
\newcommand{\pp}{pp\xspace}
\newcommand{\MeVc}{MeV/$c$\xspace}
\newcommand{\GeVc}{GeV/$c$\xspace}
\newcommand{\vtwo}{$\nu_2$\xspace}
\newcommand{\vn}{$\nu_n$\xspace}
% \newcommand{\pi0}{$\pi^0$\xspace}
\newcommand{\mum}{$\mu$m\xspace}
\newcommand{\dEdx}{$\mbox{d}E/\mbox{d}x$\xspace}
\newcommand{\RConv}{R_{\mbox{\tiny conv}}}
\newcommand{\ZConv}{Z_{\mbox{\tiny conv}}}
\newcommand{\XConv}{X_{\mbox{\tiny conv}}}
\newcommand{\YConv}{Y_{\mbox{\tiny conv}}}
\newcommand{\PhiConv}{\phi_{\mbox{\tiny conv}}}
\newcommand{\RSec}{R_{\mbox{\tiny sec}}}
\newcommand{\ZSec}{Z_{\mbox{\tiny sec}}}
\newcommand{\XSec}{X_{\mbox{\tiny sec}}}
\newcommand{\YSec}{Y_{\mbox{\tiny sec}}}
\newcommand{\PhiSec}{\phi_{\mbox{\tiny sec}}}
\newcommand{\dcaZ}{$dca_z$\xspace}
\newcommand{\pTtrack}{p_{\mbox{\tiny T},\mbox{\tiny track}}}
\newcommand{\EtaToPi}{$\eta/\pi^0$\xspace}
\newcommand{\Etat}{\eta_{\mbox{\tiny track, V0}}}
\newcommand{\EtaV}{\eta_{\mbox{\tiny V0}}}


\newcommand{\gsim}        {\stackrel{>}{\sim}}
\newcommand{\lsim}        {\stackrel{<}{\sim}}
\newcommand{\stat}        {({\mathit{ stat.}})}
\newcommand{\syst}        {({\mathit{ sys.}})}
\newcommand{\com}[1]      {}
\newcommand{\PCM}         {\acs{PCM}}
\newcommand{\PCMEMC}      {\acs{PCM}-EMC}
\newcommand{\PCMPHOS}     {\acs{PCM}-\acs{PHOS}}
\newcommand{\PCMDal}      {\acs{PCM}-Dal}
\newcommand{\EMC}         {EMC}
\newcommand{\mEMC}        {mEMC}
\newcommand{\PHOS}        {\acs{PHOS}}
\newcommand{\PE}          {\mbox{P-E}}

\setlength{\textfloatsep}{1em}

\begin{document}
%\begin{frontmatter}
%\begin{titlepage}
\title{Analysis Note:\\Neutral meson measurements with PCM-EMC and EMC \\
in ALICE in pp collisions at $\sqrt{s}$~=~13~TeV}
\ShortTitle{ALICE-ANA-XXXX}


\begin{Authlist}
H.~Bossi \Iref{a5},
J.~K\"onig\Iref{a2},
J.~L\"uhder\Iref{a1},
F.~Bock\Iref{a3},
D.~M\"uhlheim\Iref{a1},
N.~Schmidt\Iref{a6},
J.P.~Wessels\Iref{a1},
H.~B\"usching\Iref{a2},\\
C.~Loizides\Iref{a6},
for the ALICE Collaboration \Iref{a4}
\end{Authlist}

\ShortAuthor{ALICE Analysis Note ANA-XXX}
\Instfoot{a5}{Yale University, New Haven, United States}
\Instfoot{a2}{Institut f\"ur Kernphysik, Universit\"at Frankfurt, Frankfurt, Germany}
\Instfoot{a1}{Institut f\"ur Kernphysik, Westf\"alische Wilhelms-Universit\"at M{\"u}nster, M\"unster, Germany}
\Instfoot{a3}{CERN, Geneva, Switzerland}
\Instfoot{a6}{ORNL, United States}
\Instfoot{a4}{\acs{ALICE}}


%% For running titles
% \ShortAuthor{F.~Bock \emph{et al.}}
%


\vspace{4cm}
\begin{abstract}
In this analysis note, the differential invariant cross-sections of the feed-down corrected $\pi^0$ and $\eta$ mesons for \pp collisions at \sth are presented using the combined method reconstructing one photon with the \ac{PCM} and one in the \ac{EMCal} (\PCMEMC) , which can also be referred to as ``hybrid method``, as well as the standalone EMCal (\EMC) reconstruction.

Please note that this analysis was carried out analogous to the corresponding analysis at \sei \cite{anaNoteEMC8TeV,anaNotePCMEMC8TeV} and \stw \cite{anaNotePCMEMC2760GeV,anaNoteEMC2760GeV}. Therefore, the analysis notes are based on the same template and parts of the text are very close between those notes which will not be cited explicitly in this note. Thus, it has been ensured that it is not necessary to jump between notes in order to fully understand the analysis.
\end{abstract}

% \begin{keyword}
% \PACS
% \end{keyword}
% \end{frontmatter}
%\end{titlepage}
%%%%%%%%%%%%%%%%%%%%%%%%%%%%%%%%%%%%%%%%%%%%%%%%%%%%%%%%%%%%%%%%%%%%%%%%%%%%%
\newpage

\tableofcontents

\include{PhotonDetection}
\include{NeutralMesons}



\begin{appendix}

\section{Software Version \& Input Data}
\label{sec:software}
In the following, the software versions and input files are listed which have been used to generate the results shown in this note:

\begin{enumerate}
\item[-] AliPhysics \cite{AliPhysicsPCM2015} version:
\begin{center}
Software tags:\\
data: ''''\\
MB MC:  ''''\\
JetJet MC: ''''
\end{center}

\item[-] Input data / MC:
\begin{center}
data trains: GA\_pp\_AOD  \\
MB MC trains: GA\_pp\_MC\_AOD \\
JetJet MC trains: GA\_pp\_MC\_AOD
\end{center}

\item[-] Offline software (PCM AnalysisSoftware) \cite{ConvSoftware} used:
\begin{center}
Tag ''XXXX' in master branch of quoted git repository
\end{center}

\item[-] Cocktail output (Grid running) \cite{ConvSoftware} used:
\begin{center}
Software tag ''''\\
train: MBGen\_pp
\end{center}

\end{enumerate}

\section{List of Good Runs}
\label{sec:goodRuns}


 \subsection{LHC16d anchored to: LHC17f6, LHC17f6\_extra, LHC17f6\_extra2}

 \textbf{PCM:}\\
252235, 252238, 252248, 252271, 252310, 252313, 252315, 252317, 252319, 252322, 252325, 252326, 252330, 252332, 252336, 252368, 252370, 252371, 252374, 252375\\

 \subsection{LHC16g anchored to: LHC17d17, LHC17d17\_extra, LHC17d17\_extra2}

 \textbf{EDC:}\\
254128, 254147, 254148, 254149, 254174, 254199, 254204, 254205, 254293, 254302, 254303, 254304, 254330, 254331, 254332\\

 \textbf{PCMEDC:}\\
254128, 254147, 254148, 254149, 254174, 254199, 254204, 254205, 254293, 254302, 254303, 254304, 254330, 254331, 254332\\

 \textbf{PCM:}\\
254128, 254147, 254148, 254149, 254174, 254175, 254178, 254193, 254199, 254204, 254205, 254293, 254302, 254303, 254304, 254330, 254331, 254332\\

 \textbf{PHOS:}\\
254128, 254147, 254148, 254149, 254174, 254175, 254178, 254193, 254199, 254204, 254205, 254304, 254330, 254331\\

 \subsection{LHC16h anchored to: LHC17f5, LHC17f5\_extra, LHC17f5\_extra2}

 \textbf{EDC:}\\
254604, 254606, 254608, 254621, 254629, 254630, 254632, 254640, 254644, 254646, 254648, 254649, 254651, 254652, 254653, 254654, 255248, 255249, 255251, 255252, 255253, 255255, 255256, 255275, 255276, 255350, 255351, 255352, 255418, 255419, 255420, 255421, 255440, 255463, 255465, 255466, 255467\\

 \textbf{PCMEDC:}\\
254604, 254606, 254621, 254629, 254630, 254632, 254640, 254644, 254646, 254648, 254649, 254651, 254652, 254653, 254654, 255249, 255251, 255252, 255253, 255255, 255256, 255275, 255276, 255350, 255351, 255352, 255418, 255419, 255420, 255421, 255463, 255465, 255466, 255467\\

 \textbf{PCM:}\\
254418, 254419, 254422, 254604, 254606, 254608, 254621, 254629, 254630, 254632, 254640, 254644, 254646, 254648, 254649, 254651, 254652, 254653, 254654, 255079, 255082, 255085, 255086, 255091, 255111, 255154, 255159, 255162, 255167, 255171, 255173, 255174, 255176, 255177, 255240, 255242, 255247, 255248, 255249, 255251, 255252, 255253, 255255, 255256, 255275, 255276, 255280, 255283, 255350, 255351, 255352, 255398, 255402, 255407, 255415, 255418, 255419, 255420, 255421, 255440, 255442, 255447, 255463, 255465, 255466, 255467, 255469\\

 \textbf{PHOS:}\\
255275, 254983, 254984, 255242, 254604, 254606, 254608, 255249, 255251, 255252, 255253, 255255, 255256, 255177, 254621, 255466, 255008, 255009, 255010, 254629, 254630, 254378, 255111, 255276, 254381, 255407, 254640, 255240, 255154, 255283, 254644, 255467, 254646, 255159, 254648, 254649, 254394, 254395, 254396, 254653, 254654, 255167, 255171, 255173, 255176, 254476, 255180, 255182, 255280, 255440, 255442, 254419, 254422, 255447, 254479, 255162, 255248, 255398, 255079, 255465, 255082, 255463, 255085, 255086, 255247, 255091, 255350, 255351, 255402, 255011, 255415, 255418, 255419, 255420\\

 \subsection{LHC16i anchored to: LHC17d3, LHC17d3\_extra, LHC17d3\_extra2}

 \textbf{EDCtrigger:}\\
255539, 255540, 255541, 255542, 255543, 255577, 255582, 255583, 255591, 255592, 255614, 255615, 255616, 255617, 255618\\

 \textbf{EDC:}\\
255539, 255540, 255541, 255542, 255543, 255577, 255582, 255583, 255591, 255592, 255614, 255615, 255616, 255617, 255618\\

 \textbf{PCMEDC:}\\
255539, 255540, 255541, 255542, 255543, 255577, 255582, 255583, 255591, 255592, 255614, 255615, 255616, 255617, 255618\\

 \textbf{PCM:}\\
255539, 255540, 255541, 255542, 255543, 255577, 255582, 255583, 255591, 255592, 255614, 255615, 255616, 255617, 255618\\

 \textbf{PHOS:}\\
255615, 255614, 255592, 255591, 255582, 255577, 255543, 255542, 255540, 255538, 255537, 255535, 255534, 255533, 255583, 255539\\

 \subsection{LHC16j anchored to: LHC17e5, LHC17e5\_extra, LHC17e5\_extra2}

 \textbf{EDCtrigger:}\\
256207, 256231, 256281, 256282, 256283, 256284, 256289, 256290, 256292, 256295, 256297, 256298, 256299, 256302, 256307, 256309, 256311, 256356, 256357, 256361, 256362, 256364, 256365, 256366, 256371, 256372, 256373, 256415, 256417, 256418, 256420\\

 \textbf{EDC:}\\
256207, 256223, 256225, 256231, 256281, 256282, 256283, 256284, 256289, 256290, 256292, 256295, 256297, 256298, 256299, 256302, 256307, 256309, 256311, 256356, 256357, 256361, 256362, 256363, 256364, 256365, 256366, 256371, 256372, 256373, 256415, 256417, 256418, 256420 \\

 \textbf{PCMEDC:}\\
256207, 256223, 256225, 256231, 256281, 256282, 256283, 256284, 256289, 256290, 256292, 256295, 256297, 256298, 256299, 256302, 256307, 256309, 256311, 256356, 256357, 256361, 256362, 256364, 256365, 256366, 256371, 256372, 256373, 256415, 256417, 256418, 256420\\

 \textbf{PCM:}\\
256204, 256207, 256210, 256212, 256213, 256215, 256219, 256222, 256223, 256225, 256227, 256228, 256231, 256281, 256282, 256283, 256284, 256287, 256289, 256290, 256292, 256295, 256297, 256298, 256299, 256302, 256307, 256309, 256311, 256356, 256357, 256361, 256362, 256363, 256364, 256365, 256366, 256368, 256371, 256372, 256373, 256415, 256417, 256418, 256420\\

 \textbf{PHOS:}\\
256219, 256223, 256225, 256227, 256228, 256231, 256283, 256284, 256289, 256290, 256292, 256297, 256299, 256302, 256307, 256309, 256311, 256357, 256363, 256364, 256365, 256366, 256368, 256371, 256415, 256210, 256212, 256213, 256215, 256282, 256287, 256361, 256362, 256417, 256418\\

 \subsection{LHC16k anchored to: LHC18f1, LHC18f1\_extra}

 \textbf{EDCtrigger:}\\
258537, 258499, 258498, 258477, 258456, 258454, 258426, 258393, 258388, 258387, 258359, 258336, 258299, 258280, 258278, 258274, 258273, 258271, 258270, 258258, 258257, 258256, 258204, 258203, 258202, 258198, 258197, 258178, 258117, 258114, 258113, 258109, 258108, 258107, 258063, 258062, 258059, 258049, 258048, 258045, 258042, 258019, 258017, 258014, 258012, 257963, 257960, 257958, 257957, 257939, 257937, 257936, 257893, 257855, 257850, 257803, 257800, 257799, 257798, 257797, 257773, 257765, 257754, 257737, 257735, 257734, 257733, 257724, 257697, 257694, 257692, 257691, 257689, 257687, 257682, 257642, 257606, 257605, 257594, 257590, 257587, 257566, 257562, 257561, 257560, 257541, 257540, 257539, 257537, 257531, 257530, 257492, 257491, 257490, 257487, 257474, 257457, 257260, 257224, 257209, 257206, 257204, 257145, 257144, 257142, 257141, 257140, 257139, 257138, 257137, 257136, 257100, 257092, 257084, 257083, 257082, 257080, 257077, 257026, 257021, 257012, 257011, 256944, 256942, 256941, 256697, 256695, 256694, 256691, 256684, 256681, 256677, 256676, 256658, 256620, 256619, 256591, 256589, 256567, 256565, 256564, 256562, 256561, 256560, 256556, 256554, 256552, 256514, 256512, 256510, 256506, 256504\\

 \textbf{EDC:}\\
256504, 256506, 256510, 256512, 256552, 256554, 256556, 256560, 256561, 256562, 256564, 256565, 256567, 256589, 256619, 256620, 256658, 256676, 256677, 256681, 256684, 256691, 256694, 256695, 256697, 256941, 256942, 256944, 257012, 257021, 257026, 257071, 257077, 257080, 257082, 257083, 257084, 257092, 257100, 257136, 257137, 257138, 257139, 257140, 257141, 257142, 257144, 257145, 257204, 257206, 257209, 257224, 257260, 257318, 257320, 257322, 257364, 257433, 257457, 257474, 257487, 257490, 257491, 257492, 257530, 257531, 257537, 257539, 257540, 257541, 257560, 257561, 257562, 257566, 257587, 257590, 257594, 257605, 257606, 257630, 257642, 257682, 257684, 257687, 257688, 257689, 257691, 257692, 257694, 257697, 257724, 257725, 257733, 257734, 257735, 257737, 257754, 257765, 257773, 257797, 257798, 257799, 257800, 257803, 257804, 257850, 257853, 257855, 257893, 257936, 257937, 257939, 257957, 257958, 257960, 257963, 257979, 258012, 258014, 258017, 258019, 258039, 258041, 258042, 258045, 258048, 258049, 258059, 258060, 258062, 258063, 258107, 258108, 258109, 258113, 258114, 258117, 258178, 258197, 258198, 258202, 258203, 258204, 258256, 258257, 258258, 258270, 258271, 258273, 258274, 258278, 258299, 258336, 258359, 258393, 258426, 258454, 258456, 258477, 258499, 258537\\

 \textbf{PCMEDC:}\\
258537, 258499, 258477, 258456, 258454, 258426, 258393, 258359, 258336, 258299, 258278, 258274, 258273, 258271, 258270, 258258, 258257, 258256, 258204, 258203, 258202, 258198, 258197, 258178, 258117, 258114, 258113, 258109, 258108, 258107, 258063, 258062, 258059, 258049, 258048, 258045, 258042, 258019, 258017, 258014, 258012, 257963, 257960, 257958, 257957, 257939, 257937, 257936, 257893, 257855, 257850, 257803, 257800, 257799, 257798, 257797, 257773, 257765, 257754, 257737, 257735, 257734, 257733, 257724, 257697, 257694, 257692, 257691, 257689, 257687, 257682, 257642, 257606, 257605, 257594, 257590, 257587, 257566, 257562, 257561, 257560, 257541, 257540, 257539, 257537, 257531, 257530, 257492, 257491, 257490, 257487, 257474, 257457, 257433, 257364, 257322, 257320, 257318, 257260, 257224, 257209, 257206, 257204, 257145, 257144, 257142, 257141, 257140, 257139, 257138, 257137, 257136, 257100, 257092, 257084, 257083, 257082, 257080, 257077, 257026, 257021, 257012, 256944, 256942, 256941, 256697, 256695, 256694, 256691, 256684, 256681, 256677, 256676, 256658, 256620, 256619, 256589, 256567, 256565, 256564, 256562, 256561, 256560, 256556, 256554, 256552, 256512, 256510, 256506, 256504\\

 \textbf{PCM:}\\
256504, 256506, 256510, 256512, 256552, 256554, 256556, 256557, 256560, 256561, 256562, 256564, 256565, 256567, 256589, 256619, 256620, 256658, 256676, 256677, 256681, 256684, 256691, 256692, 256694, 256695, 256697, 256941, 256942, 256944, 257012, 257021, 257026, 257071, 257077, 257080, 257082, 257083, 257084, 257086, 257092, 257095, 257100, 257136, 257137, 257138, 257139, 257140, 257141, 257142, 257144, 257145, 257204, 257206, 257209, 257224, 257260, 257318, 257320, 257322, 257330, 257358, 257364, 257433, 257457, 257468, 257474, 257487, 257488, 257490, 257491, 257492, 257530, 257531, 257537, 257539, 257540, 257541, 257560, 257561, 257562, 257566, 257587, 257588, 257590, 257592, 257594, 257595, 257604, 257605, 257606, 257630, 257632, 257635, 257636, 257642, 257644, 257682, 257684, 257685, 257687, 257688, 257689, 257691, 257692, 257694, 257697, 257724, 257725, 257727, 257733, 257734, 257735, 257737, 257754, 257757, 257765, 257773, 257797, 257798, 257799, 257800, 257803, 257804, 257850, 257851, 257853, 257855, 257893, 257901, 257912, 257932, 257936, 257937, 257939, 257957, 257958, 257960, 257963, 257979, 257986, 257989, 257992, 258003, 258008, 258012, 258014, 258017, 258019, 258039, 258041, 258042, 258045, 258048, 258049, 258053, 258059, 258060, 258062, 258063, 258107, 258108, 258109, 258113, 258114, 258117, 258178, 258197, 258198, 258202, 258203, 258204, 258256, 258257, 258258, 258270, 258271, 258273, 258274, 258278, 258299, 258301, 258302, 258303, 258306, 258307, 258332, 258336, 258359, 258391, 258393, 258426, 258452, 258454, 258456, 258477, 258499, 258537\\

 \textbf{PHOS:}\\
257682, 257684, 257685, 257687, 257688, 257689, 257692, 257697, 257724, 257725, 257727, 257733, 257734, 257735, 257737, 257754, 257757, 257765, 257773, 257797, 257798, 257799, 257800, 257803, 257804, 257850, 257851, 257853, 257893, 257901, 257912, 257932, 257936, 257937, 257939, 257957, 257958, 257960, 257963, 257986, 257989, 257992, 258003, 258008, 258019, 258039, 258041, 258045, 258048, 258049, 258053, 258059, 258060, 258062, 258063, 258107, 258108, 258109, 258113, 258114, 258117, 258178, 258197, 258198, 258202, 258203, 258204, 258256, 258257, 258258, 258270, 258271, 258274, 258278, 258280, 258301, 258302, 258303, 258307, 258332, 258336, 258359, 258388, 258391, 258393, 258452, 258454, 258456, 258477, 258498, 258499\\

 \subsection{LHC16l anchored to: LHC18d8, LHC18d8\_extra}

 \textbf{EDC:}\\
258962, 258964, 259088, 259090, 259091, 259096, 259099, 259117, 259118, 259164, 259204, 259257, 259263, 259269, 259270, 259271, 259272, 259273, 259274, 259302, 259305, 259307, 259334, 259336, 259339, 259340, 259341, 259342, 259378, 259382, 259388, 259389, 259394, 259395, 259396, 259473, 259477, 259649, 259650, 259668, 259697, 259703, 259704, 259711, 259747, 259748, 259750, 259751, 259752, 259756, 259781, 259788, 259822, 259841, 259842, 259860, 259866, 259867, 259868, 259888\\

 \textbf{PCMEDC:}\\
258962, 258964, 259088, 259090, 259091, 259096, 259099, 259117, 259118, 259164, 259204, 259257, 259263, 259269, 259270, 259271, 259272, 259273, 259274, 259302, 259305, 259307, 259334, 259336, 259339, 259340, 259341, 259342, 259378, 259382, 259388, 259389, 259394, 259395, 259396, 259473, 259477, 259649, 259650, 259668, 259697, 259703, 259704, 259711, 259747, 259748, 259750, 259751, 259752, 259756, 259781, 259788, 259822, 259841, 259842, 259860, 259866, 259867, 259868, 259888\\

 \textbf{PCM:}\\
258962, 258964, 259086, 259088, 259090, 259091, 259096, 259099, 259117, 259118, 259162, 259164, 259204, 259257, 259261, 259263, 259264, 259269, 259270, 259271, 259272, 259273, 259274, 259302, 259303, 259305, 259307, 259334, 259336, 259339, 259340, 259341, 259342, 259378, 259381, 259382, 259388, 259389, 259394, 259395, 259396, 259473, 259477, 259649, 259650, 259668, 259697, 259700, 259703, 259704, 259705, 259711, 259713, 259747, 259748, 259750, 259751, 259752, 259756, 259781, 259788, 259789, 259822, 259841, 259842, 259860, 259866, 259867, 259868, 259888\\

 \textbf{PHOS:}\\
258883, 258884, 258885, 258886, 258889, 258890, 258919, 258920, 258921, 258923, 258926, 258931, 258962, 258964, 259086, 259088, 259090, 259096, 259099, 259117, 259162, 259164, 259204, 259261, 259263, 259264, 259269, 259270, 259271, 259272, 259273, 259274, 259302, 259303, 259305, 259307, 259334, 259336, 259339, 259340, 259341, 259342, 259378, 259381, 259389, 259394, 259395, 259473, 259477, 259649, 259650, 259668, 259697, 259703, 259704, 259705, 259711, 259713, 259747, 259750, 259751, 259752, 259756, 259788, 259789, 259822, 259841, 259842, 259860, 259866, 259867, 259868, 259888, 259961, 259979, 260010, 260011\\

 \subsection{LHC16o anchored to: LHC17d16, LHC17d16\_extra, LHC17d16\_extra2}

 \textbf{EDCtrigger:}\\
264035, 264033, 263985, 263984, 263981, 263979, 263978, 263977, 263923, 263917, 263916, 263905, 263866, 263863, 263861, 263803, 263793, 263792, 263790, 263786, 263785, 263784, 263744, 263743, 263741, 263739, 263738, 263737, 263691, 263690, 263689, 263682, 263663, 263662, 263657, 263654, 263653, 263652, 262778, 262777, 262776, 262768, 262760, 262727, 262719, 262717, 262713, 262708, 262706, 262705, 262635, 262632, 262628, 262624, 262594, 262593, 262583, 262574, 262572, 262571, 262570, 262569, 262567, 262563, 262533, 262532, 262492, 262451, 262450\\

 \textbf{EDC:}\\
262450, 262451, 262492, 262528, 262532, 262533, 262563, 262567, 262569, 262570, 262571, 262572, 262574, 262583, 262593, 262594, 262624, 262628, 262632, 262635, 262705, 262706, 262708, 262713, 262717, 262719, 262723, 262727, 262760, 262768, 262776, 262777, 262778, 262841, 262844, 262849, 262853, 262858, 263332, 263487, 263496, 263529, 263647, 263652, 263653, 263654, 263657, 263662, 263663, 263682, 263689, 263690, 263691, 263737, 263738, 263739, 263741, 263743, 263744, 263784, 263785, 263786, 263787, 263790, 263792, 263793, 263803, 263861, 263863, 263866, 263905, 263916, 263917, 263977, 263978, 263979, 263981, 263984, 263985, 264033, 264035\\

 \textbf{PCMEDC:}\\
264035, 264033, 263985, 263984, 263981, 263979, 263978, 263977, 263917, 263916, 263905, 263866, 263863, 263861, 263803, 263793, 263792, 263790, 263787, 263786, 263785, 263784, 263744, 263743, 263741, 263739, 263738, 263737, 263691, 263690, 263689, 263682, 263663, 263662, 263657, 263654, 263653, 263652, 263496, 263487, 263332, 262858, 262853, 262849, 262844, 262841, 262778, 262777, 262776, 262768, 262760, 262727, 262719, 262717, 262713, 262708, 262706, 262705, 262635, 262632, 262628, 262624, 262594, 262593, 262583, 262574, 262572, 262571, 262570, 262569, 262567, 262563, 262533, 262532, 262492, 262451, 262450\\

 \textbf{PCM:}\\
262450, 262451, 262487, 262489, 262490, 262492, 262528, 262532, 262533, 262537, 262563, 262567, 262568, 262569, 262570, 262571, 262572, 262574, 262578, 262583, 262593, 262594, 262624, 262628, 262632, 262635, 262705, 262706, 262708, 262713, 262717, 262719, 262723, 262725, 262727, 262760, 262768, 262776, 262777, 262778, 262841, 262842, 262844, 262847, 262849, 262853, 262855, 262858, 263332, 263487, 263490, 263496, 263497, 263529, 263647, 263652, 263653, 263654, 263657, 263662, 263663, 263682, 263689, 263690, 263691, 263737, 263738, 263739, 263741, 263743, 263744, 263784, 263785, 263786, 263787, 263790, 263792, 263793, 263803, 263810, 263861, 263863, 263866, 263905, 263916, 263917, 263920, 263977, 263978, 263979, 263981, 263984, 263985, 264033, 264035\\

 \textbf{PHOS:}\\
262395, 262396, 262397, 262398, 262399, 262418, 262419, 262422, 262423, 262424, 262425, 262426, 262428, 262430, 262450, 262451, 262487, 262489, 262490, 262492, 262528, 262532, 262533, 262537, 262563, 262567, 262568, 262569, 262570, 262571, 262572, 262574, 262578, 262583, 262593, 262624, 262628, 262632, 262635, 262705, 262706, 262708, 262713, 262717, 262723, 262725, 262727, 262768, 262776, 262777, 262778, 262842, 262844, 262847, 262849, 262853, 262855, 263487, 263490, 263497, 263529, 263647, 263652, 263654, 263657, 263662, 263663, 263682, 263689, 263690, 263691, 263737, 263739, 263741, 263743, 263784, 263785, 263786, 263787, 263790, 263792, 263803, 263810, 263813, 263823, 263824, 263829, 263830, 263861, 263863, 263866, 263905, 263916, 263917, 263920, 263977, 263978, 263979, 263981, 263984, 263985\\

 \subsection{LHC16p anchored to: LHC17d18, LHC17d18\_extra, LHC17d18\_extra2}

 \textbf{EDCtrigger:}\\
264347, 264346, 264345, 264336, 264312, 264306, 264305, 264281, 264279, 264277, 264267, 264266, 264265, 264264, 264262, 264261, 264260, 264259, 264238, 264235, 264233, 264232, 264198, 264197, 264190, 264188, 264168, 264164, 264139, 264138, 264137, 264129, 264110, 264109, 264086, 264085, 264082, 264078, 264076\\

 \textbf{EDC:}\\
264347, 264346, 264345, 264336, 264312, 264306, 264305, 264281, 264279, 264277, 264267, 264266, 264265, 264264, 264262, 264261, 264260, 264259, 264238, 264235, 264233, 264232, 264198, 264197, 264190, 264188, 264168, 264164, 264139, 264138, 264137, 264129, 264110, 264109, 264086, 264085, 264082, 264078, 264076\\

 \textbf{PCMEDC:}\\
264347, 264346, 264345, 264336, 264312, 264306, 264305, 264281, 264279, 264277, 264267, 264266, 264265, 264264, 264262, 264261, 264260, 264259, 264238, 264235, 264233, 264232, 264198, 264197, 264190, 264188, 264168, 264164, 264139, 264138, 264137, 264129, 264110, 264109, 264086, 264085, 264082, 264078, 264076\\

 \textbf{PCM:}\\
264076, 264078, 264082, 264085, 264086, 264109, 264110, 264129, 264137, 264138, 264139, 264164, 264188, 264190, 264194, 264197, 264198, 264232, 264233, 264235, 264238, 264259, 264260, 264261, 264262, 264264, 264265, 264266, 264267, 264273, 264277, 264279, 264281, 264305, 264306, 264312, 264336, 264341, 264345, 264346, 264347\\

 \textbf{PHOS:}\\
264076, 264078, 264082, 264085, 264086, 264109, 264110, 264129, 264137, 264138, 264139, 264164, 264168, 264188, 264190, 264197, 264198, 264232, 264233, 264235, 264238, 264259, 264260, 264261, 264262, 264264, 264265, 264266, 264267, 264277, 264279, 264281, 264306, 264312, 264336, 264341, 264345, 264346\\

 \subsection{LHC17c anchored to: LHC18d3, LHC18d3\_extra}

 \textbf{EDC:}\\
270661, 270663, 270665\\

 \textbf{PCMEDC:}\\
270661, 270663, 270665\\

 \textbf{PCM:}\\
270581, 270661, 270663, 270665, 270667\\

 \textbf{PHOS:}\\
270667, 270665, 270663, 270661, 270581\\

 \subsection{LHC17e anchored to:  }

 \textbf{PCM:}\\
270822, 270824, 270827, 270828, 270830\\

 \textbf{PHOS:}\\
270830, 270828, 270827, 270822\\

 \subsection{LHC17f anchored to:  }

 \textbf{PCM:}\\
270854, 270855, 270856, 270861, 270865\\

 \textbf{PHOS:}\\
270854, 270855, 270856, 270861, 270865\\

 \subsection{LHC17h anchored to: LHC18c12, LHC18c12\_extra}

 \textbf{EDCtrigger:}\\
272076, 272100, 272101, 272123, 272151, 272152, 272153, 272154, 272155, 272156, 272194, 272335, 272340, 272359, 272360, 272388, 272389, 272394, 272395, 272399, 272400, 272411, 272413, 272414, 272417, 272461, 272462, 272463, 272466, 272468, 272469, 272521, 272574, 272575, 272585, 272607, 272608, 272610, 272620, 272690, 272691, 272692, 272712, 272746, 272747, 272749, 272760, 272762, 272763, 272764, 272782, 272783, 272828, 272903, 272905, 272932, 272933, 272934, 272947, 272949, 273009, 273077, 273099, 273100\\

 \textbf{EDC:}\\
271870, 271871, 271873, 271874, 271880, 271886, 272018, 272020, 272076, 272100, 272101, 272123, 272151, 272152, 272153, 272154, 272155, 272156, 272194, 272335, 272340, 272359, 272360, 272388, 272389, 272394, 272395, 272399, 272400, 272411, 272413, 272461, 272462, 272463, 272466, 272468, 272521, 272574, 272575, 272585, 272607, 272608, 272610, 272620, 272690, 272691, 272712, 272747, 272749, 272760, 272763, 272764, 272782, 272783, 272784, 272828, 272903, 272905, 272932, 272933, 272934, 272935, 272947, 272949, 273009, 273077, 273099, 273100, 272949, 273009, 273077, 273099, 273100\\

 \textbf{PCMEDC:}\\
271868, 271870, 271871, 271873, 271874, 271880, 271881, 271886, 272018, 272020, 272076, 272100, 272101, 272123, 272151, 272152, 272153, 272154, 272155, 272156, 272194, 272335, 272340, 272359, 272360, 272388, 272389, 272394, 272395, 272399, 272400, 272411, 272413, 272414, 272417, 272461, 272462, 272463, 272466, 272468, 272469, 272521, 272574, 272575, 272585, 272607, 272608, 272610, 272620, 272690, 272691, 272692, 272712, 272746, 272747, 272749, 272760, 272762, 272763, 272764, 272782, 272783, 272784, 272828, 272903, 272905, 272932, 272933, 272934, 272947, 272949, 273009, 273077, 273099, 273100\\

 \textbf{PCM:}\\
273103, 273100, 273099, 273077, 273010, 273009, 272985, 272983, 272976, 272949, 272947, 272939, 272935, 272934, 272933, 272932, 272905, 272903, 272880, 272873, 272871, 272870, 272836, 272834, 272833, 272829, 272828, 272784, 272783, 272782, 272764, 272763, 272760, 272749, 272747, 272712, 272691, 272690, 272620, 272610, 272608, 272607, 272585, 272577, 272575, 272574, 272521, 272468, 272466, 272463, 272462, 272461, 272413, 272411, 272400, 272399, 272395, 272394, 272389, 272388, 272360, 272359, 272340, 272335, 272194, 272156, 272155, 272154, 272153, 272152, 272151, 272123, 272101, 272100, 272076, 272042, 272040, 272039, 272038, 272036, 272020, 272018, 271886, 271880, 271874, 271873, 271871, 271870\\

 \textbf{PHOS:}\\
272151, 272152, 272153, 272154, 272155, 272156, 272194, 272335, 272340, 272359, 272360, 272388, 272389, 272394, 272395, 272399, 272400, 272411, 272413, 272414, 272417, 272461, 272462, 272463, 272466, 272468, 272469, 272521, 272574, 272575, 272577, 272585, 272607, 272608, 272610, 272620, 272690, 272691, 272692, 272712, 272746, 272747, 272749, 272760, 272762, 272763, 272764, 272782, 272783, 272784, 272828, 272829, 272833, 272834, 272836, 272870, 272871, 272880, 272903, 272905, 272932, 272933, 272934, 272935, 272947, 272949, 272976, 272983, 272985, 273009, 273010, 273077, 273099, 273100, 273101, 273103\\

 \subsection{LHC17i anchored to: LHC17k4, LHC17k4\_extra}

 \textbf{EDCtrigger:}\\
273824, 273825, 273885, 273886, 273887, 273889, 273918, 273942, 273943, 273946, 273985, 273986, 274063, 274064, 274092, 274125, 274147, 274148, 274174, 274212, 274232, 274258, 274259, 274263, 274264, 274269, 274270, 274271, 274278, 274280, 274281, 274283, 274329, 274351, 274352, 274355, 274363, 274364, 274442\\

 \textbf{EDC:}\\
273591, 273592, 273593, 273653, 273654, 273687, 273689, 273690, 273695, 273709, 273711, 273719, 273824, 273825, 273885, 273886, 273887, 273889, 273918, 273942, 273943, 273946, 273985, 273986, 274063, 274064, 274092, 274125, 274147, 274148, 274174, 274212, 274232, 274258, 274259, 274263, 274264, 274269, 274270, 274271, 274278, 274280, 274281, 274283, 274329, 274351, 274352, 274355, 274363, 274364, 274385, 274386, 274387, 274388, 274389, 274390, 274442\\

 \textbf{PCMEDC:}\\
273591, 273592, 273593, 273653, 273654, 273687, 273689, 273690, 273695, 273709, 273711, 273719, 273824, 273825, 273885, 273886, 273887, 273889, 273918, 273942, 273943, 273946, 273985, 273986, 274064, 274092, 274125, 274147, 274148, 274174, 274212, 274232, 274258, 274259, 274263, 274264, 274269, 274270, 274271, 274278, 274280, 274281, 274283, 274329, 274351, 274352, 274363, 274364, 274385, 274386, 274387, 274388, 274389, 274390, 274442\\

 \textbf{PCM:}\\
273591, 273592, 273593, 273653, 273654, 273687, 273689, 273690, 273695, 273709, 273711, 273719, 273824, 273825, 273885, 273886, 273887, 273889, 273918, 273942, 273943, 273946, 273985, 273986, 274058, 274064, 274092, 274125, 274147, 274148, 274174, 274212, 274232, 274258, 274259, 274263, 274264, 274266, 274268, 274269, 274270, 274271, 274276, 274278, 274280, 274281, 274283, 274329, 274351, 274352, 274360, 274363, 274364, 274385, 274386, 274387, 274388, 274389, 274390, 274442\\

 \textbf{PHOS:}\\
273824, 273825, 273885, 273886, 273887, 273889, 273918, 273942, 273943, 273946, 273985, 273986, 274058, 274063, 274064, 274092, 274094, 274125, 274147, 274148, 274174, 274212, 274232, 274258, 274259, 274263, 274264, 274266, 274268, 274269, 274270, 274271, 274276, 274278, 274279, 274280, 274281, 274283, 274329, 274351, 274352, 274355, 274357, 274360, 274363, 274364, 274442\\

 \subsection{LHC17j anchored to: LHC17h11, LHC17h11\_extra}

 \textbf{EDCtrigger:}\\
274593, 274595, 274596, 274601, 274653, 274657, 274667, 274669, 274671\\

 \textbf{EDC:}\\
274593, 274595, 274596, 274601, 274653, 274657, 274667, 274669, 274671\\

 \textbf{PCMEDC:}\\
274593, 274595, 274596, 274601, 274653, 274657, 274667, 274669, 274671\\

 \textbf{PCM:}\\
274593, 274594, 274595, 274596, 274601, 274653, 274657, 274667, 274669, 274671\\

 \textbf{PHOS:}\\
274593, 274594, 274595, 274596, 274601, 274653, 274657, 274667, 274669, 274671\\

 \subsection{LHC17k anchored to: LHC18c13, LHC18c13\_extra}

 \textbf{EDCtrigger:}\\
274708, 274801, 274802, 274806, 274815, 274821, 274822, 274877, 274878, 274882, 275067, 275068, 275073, 275075, 275076, 275149, 275150, 275151, 275173, 275174, 275177, 275239, 275245, 275246, 275247, 275283, 275314, 275324, 275326, 275328, 275332, 275360, 275361, 275369, 275372, 275401, 275406, 275448, 275452, 275453, 275456, 275457, 275471, 275472, 275515, 275558, 275559, 275612, 275621, 275622, 275623, 275624, 275648, 275661, 275664, 276097, 276098, 276102, 276104, 276135, 276140, 276145, 276166, 276169, 276170, 276178, 276205, 276230, 276257, 276259, 276290, 276292, 276294, 276302, 276348, 276351, 276435, 276437, 276438, 276439, 276462, 276506, 276507, 276508\\

 \textbf{EDC:}\\
274690, 274708, 274801, 274802, 274806, 274815, 274821, 274822, 274877, 274878, 274882, 275067, 275068, 275073, 275075, 275076, 275149, 275150, 275151, 275173, 275174, 275177, 275239, 275245, 275246, 275247, 275283, 275314, 275324, 275326, 275328, 275332, 275360, 275361, 275369, 275372, 275401, 275406, 275443, 275448, 275452, 275453, 275456, 275457, 275471, 275472, 275515, 275558, 275559, 275612, 275621, 275622, 275623, 275624, 275648, 275661, 275664, 275847, 276097, 276098, 276102, 276104, 276135, 276140, 276145, 276166, 276169, 276170, 276178, 276205, 276230, 276257, 276259, 276290, 276292, 276294, 276302, 276348, 276351, 276435, 276437, 276438, 276439, 276462, 276506, 276507, 276508\\

 \textbf{PCMEDC:}\\
274690, 274708, 274801, 274802, 274806, 274815, 274821, 274822, 274877, 274878, 274882, 275067, 275068, 275073, 275075, 275076, 275149, 275150, 275151, 275173, 275174, 275177, 275239, 275245, 275246, 275247, 275283, 275314, 275324, 275326, 275328, 275332, 275360, 275361, 275369, 275372, 275401, 275406, 275443, 275448, 275452, 275453, 275456, 275457, 275471, 275472, 275515, 275558, 275559, 275612, 275621, 275622, 275623, 275624, 275648, 275661, 275664, 275847, 276097, 276098, 276102, 276104, 276135, 276140, 276145, 276166, 276169, 276170, 276178, 276205, 276230, 276257, 276259, 276290, 276292, 276294, 276302, 276348, 276351, 276435, 276437, 276438, 276439, 276462, 276506, 276507, 276508\\

 \textbf{PCM:}\\
274690, 274708, 274801, 274802, 274803, 274806, 274807, 274811, 274815, 274817, 274821, 274822, 274877, 274878, 274882, 274886, 274978, 274979, 275067, 275068, 275073, 275075, 275076, 275149, 275150, 275151, 275173, 275174, 275177, 275180, 275184, 275188, 275239, 275245, 275246, 275247, 275283, 275314, 275322, 275324, 275326, 275328, 275332, 275333, 275360, 275361, 275369, 275372, 275394, 275395, 275401, 275404, 275406, 275443, 275448, 275452, 275453, 275456, 275457, 275459, 275467, 275471, 275472, 275515, 275558, 275559, 275612, 275617, 275621, 275622, 275623, 275624, 275647, 275648, 275650, 275657, 275661, 275664, 275847, 275924, 275925, 276012, 276013, 276017, 276019, 276020, 276040, 276041, 276045, 276097, 276098, 276099, 276102, 276104, 276135, 276140, 276145, 276166, 276169, 276170, 276177, 276178, 276205, 276230, 276257, 276259, 276290, 276291, 276292, 276294, 276297, 276302, 276307, 276312, 276348, 276351, 276435, 276437, 276438, 276439, 276462, 276506, 276507, 276508\\

 \textbf{PHOS:}\\
274736, 274801, 274802, 274803, 274806, 274807, 274811, 274815, 274817, 274821, 274822, 274877, 274878, 274882, 274883, 274884, 274886, 274889, 274978, 274979, 275067, 275068, 275073, 275075, 275076, 275149, 275150, 275151, 275173, 275174, 275177, 275180, 275184, 275188, 275245, 275246, 275247, 275283, 275314, 275322, 275324, 275326, 275328, 275332, 275333, 275360, 275361, 275369, 275372, 275394, 275395, 275401, 275404, 275406, 275443, 275448, 275452, 275453, 275456, 275457, 275459, 275467, 275471, 275472, 275515, 275558, 275559, 275612, 275617, 275621, 275622, 275623, 275624, 275647, 275648, 275650, 275657, 275661, 275664, 275924, 275925, 276040, 276041, 276045, 276097, 276098, 276099, 276102, 276104, 276105, 276108, 276135, 276140, 276141, 276145, 276166, 276169, 276170, 276177, 276178, 276205, 276230, 276257, 276259, 276290, 276291, 276292, 276294, 276297, 276302, 276307, 276348, 276351, 276435, 276438, 276439, 276462, 276506, 276507, 276508\\

 \subsection{LHC17l anchored to: LHC18a8, LHC18a8\_extra}

 \textbf{EDCtrigger:}\\
276552, 276553, 276556, 276557, 276608, 276669, 276670, 276671, 276672, 276674, 276675, 276762, 276916, 276917, 276920, 276970, 276971, 276972, 277015, 277016, 277017, 277037, 277073, 277076, 277079, 277087, 277091, 277121, 277155, 277180, 277181, 277182, 277183, 277188, 277189, 277193, 277194, 277196, 277197, 277256, 277257, 277262, 277293, 277310, 277312, 277314, 277360, 277383, 277384, 277385, 277389, 277472, 277473, 277476, 277477, 277478, 277479, 277530, 277536, 277537, 277574, 277575, 277576, 277577, 277721, 277722, 277723, 277725, 277745, 277746, 277747, 277749, 277794, 277799, 277800, 277801, 277802, 277834, 277842, 277845, 277847, 277848, 277876, 277897, 277898, 277899, 277900, 277901, 277903, 277907, 277930, 277952, 277987, 277988, 277989, 277996, 278121, 278122, 278126, 278127, 278158, 278163, 278164, 278165, 278166, 278167, 278189, 278191, 278215\\

 \textbf{EDC:}\\
276551, 276552, 276553, 276556, 276557, 276608, 276644, 276670, 276671, 276672, 276674, 276675, 276762, 276916, 276917, 276920, 276967, 276969, 276970, 276971, 276972, 277015, 277016, 277017, 277037, 277073, 277076, 277079, 277087, 277091, 277121, 277155, 277180, 277181, 277182, 277183, 277188, 277189, 277193, 277194, 277196, 277197, 277256, 277257, 277262, 277293, 277310, 277312, 277314, 277360, 277383, 277384, 277385, 277389, 277416, 277417, 277418, 277472, 277473, 277476, 277477, 277478, 277479, 277530, 277531, 277534, 277536, 277537, 277574, 277575, 277576, 277577, 277721, 277722, 277723, 277725, 277745, 277746, 277747, 277749, 277794, 277799, 277800, 277801, 277802, 277834, 277841, 277842, 277845, 277847, 277848, 277876, 277897, 277898, 277899, 277900, 277901, 277903, 277907, 277930, 277952, 277987, 277989, 277996, 278121, 278122, 278126, 278127, 278158, 278163, 278164, 278165, 278166, 278167, 278189, 278191, 278215, 278216\\

 \textbf{PCMEDC:}\\
276552, 276553, 276556, 276557, 276608, 276644, 276669, 276670, 276671, 276672, 276674, 276675, 276762, 276916, 276917, 276920, 276967, 276969, 276970, 276971, 276972, 277015, 277016, 277017, 277037, 277073, 277076, 277079, 277087, 277091, 277121, 277155, 277180, 277181, 277182, 277183, 277188, 277189, 277193, 277194, 277196, 277197, 277256, 277257, 277262, 277293, 277310, 277312, 277314, 277360, 277383, 277384, 277385, 277389, 277416, 277417, 277418, 277470, 277472, 277473, 277476, 277477, 277478, 277479, 277530, 277531, 277534, 277536, 277537, 277574, 277575, 277576, 277577, 277721, 277722, 277723, 277725, 277745, 277746, 277747, 277749, 277794, 277799, 277800, 277801, 277802, 277834, 277841, 277842, 277845, 277847, 277848, 277876, 277897, 277898, 277899, 277900, 277901, 277903, 277907, 277930, 277952, 277987, 277988, 277989, 277996, 278121, 278122, 278126, 278127, 278158, 278163, 278164, 278165, 278166, 278167, 278189, 278191, 278215, 278216\\

 \textbf{PCM:}\\
276551, 276552, 276553, 276556, 276557, 276608, 276644, 276670, 276671, 276672, 276674, 276675, 276762, 276916, 276917, 276920, 276967, 276969, 276970, 276971, 276972, 277015, 277016, 277017, 277037, 277073, 277076, 277079, 277082, 277087, 277091, 277117, 277121, 277155, 277180, 277181, 277182, 277183, 277184, 277188, 277189, 277193, 277194, 277196, 277197, 277256, 277257, 277262, 277293, 277310, 277312, 277314, 277360, 277383, 277384, 277385, 277386, 277389, 277416, 277417, 277418, 277472, 277473, 277476, 277477, 277478, 277479, 277530, 277531, 277534, 277536, 277537, 277574, 277575, 277576, 277577, 277721, 277722, 277723, 277725, 277745, 277746, 277747, 277749, 277794, 277795, 277799, 277800, 277801, 277802, 277805, 277834, 277836, 277841, 277842, 277845, 277847, 277848, 277870, 277876, 277897, 277898, 277899, 277900, 277901, 277903, 277904, 277907, 277930, 277952, 277987, 277989, 277991, 277996, 278121, 278122, 278123, 278126, 278127, 278158, 278163, 278164, 278165, 278166, 278167, 278189, 278191, 278215, 278216\\

 \textbf{PHOS:}\\
276551, 276552, 276553, 276556, 276557, 276608, 276644, 276669, 276670, 276671, 276672, 276674, 276675, 276762, 276916, 276917, 276920, 276967, 276969, 276970, 276971, 276972, 277015, 277016, 277017, 277037, 277073, 277076, 277079, 277082, 277087, 277091, 277117, 277121, 277155, 277180, 277181, 277182, 277183, 277184, 277188, 277189, 277193, 277194, 277196, 277197, 277256, 277257, 277262, 277293, 277310, 277312, 277314, 277360, 277383, 277384, 277385, 277386, 277389, 277416, 277417, 277418, 277470, 277472, 277473, 277476, 277477, 277478, 277479, 277530, 277531, 277536, 277537, 277574, 277575, 277576, 277577, 277721, 277722, 277723, 277725, 277745, 277746, 277747, 277749, 277794, 277795, 277799, 277800, 277801, 277802, 277805, 277834, 277836, 277841, 277842, 277845, 277847, 277848, 277870, 277876, 277897, 277898, 277899, 277900, 277901, 277903, 277904, 277907, 277930, 277952, 277987, 277988, 277991, 278121, 278122, 278123, 278126, 278127, 278130, 278158, 278163, 278164, 278165, 278166, 278167, 278189, 278191, 278215, 278216\\

 \subsection{LHC17m anchored to: LHC17l5, LHC17l5\_extra}

 \textbf{EDCtrigger:}\\
 279005, 279008, 279035, 279036, 279041, 279043, 279044, 279068, 279069, 279073, 279074, 279075, 279106, 279107, 279117, 279118, 279122, 279123, 279130, 279155, 279157, 279199, 279201, 279207, 279208, 279232, 279234, 279235, 279242, 279264, 279265, 279268, 279270, 279273, 279274, 279309, 279310, 279312, 279342, 279344, 279348, 279355, 279391, 279410, 279439, 279441, 279483, 279487, 279491, 279550, 279559, 279632, 279641, 279642, 279676, 279677, 279682, 279687, 279688, 279689, 279718, 279719, 279747, 279749, 279773, 279826, 279853, 279855, 280051, 280052, 280066, 280107, 280108, 280111, 280114, 280126, 280131, 280134, 280140\\

 \textbf{EDC:}\\
278915, 278936, 278939, 278941, 278959, 278960, 278963, 278964, 278999, 279000, 279005, 279007, 279008, 279035, 279036, 279041, 279043, 279044, 279068, 279069, 279073, 279074, 279075, 279106, 279107, 279117, 279118, 279122, 279123, 279130, 279155, 279157, 279199, 279201, 279207, 279208, 279232, 279234, 279235, 279242, 279264, 279265, 279268, 279270, 279273, 279274, 279309, 279310, 279312, 279342, 279344, 279348, 279354, 279355, 279391, 279410, 279439, 279441, 279483, 279487, 279488, 279491, 279550, 279559, 279630, 279632, 279641, 279642, 279676, 279677, 279682, 279683, 279687, 279688, 279689, 279718, 279719, 279747, 279749, 279773, 279826, 279830, 279853, 279855, 280051, 280052, 280066, 280107, 280108, 280111, 280114, 280126, 280131, 280134, 280140\\

 \textbf{PCMEDC:}\\
278915, 278936, 278939, 278941, 278959, 278960, 278963, 278964, 278999, 279000, 279005, 279007, 279008, 279035, 279036, 279041, 279043, 279044, 279068, 279069, 279073, 279074, 279075, 279106, 279107, 279117, 279118, 279122, 279123, 279130, 279155, 279157, 279199, 279201, 279207, 279208, 279232, 279234, 279235, 279242, 279264, 279265, 279268, 279270, 279273, 279274, 279309, 279310, 279312, 279342, 279344, 279348, 279354, 279355, 279391, 279410, 279439, 279441, 279483, 279487, 279488, 279491, 279550, 279559, 279630, 279632, 279641, 279642, 279676, 279677, 279682, 279683, 279687, 279688, 279689, 279718, 279719, 279747, 279749, 279773, 279826, 279830, 279853, 279855, 280051, 280052, 280066, 280107, 280108, 280111, 280114, 280126, 280131, 280134, 280140\\

 \textbf{PCM:}\\
280140, 280135, 280134, 280131, 280126, 280118, 280114, 280111, 280108, 280107, 280066, 280052, 280051, 279879, 279855, 279854, 279853, 279830, 279827, 279826, 279773, 279749, 279747, 279719, 279718, 279715, 279689, 279688, 279687, 279684, 279683, 279682, 279679, 279677, 279676, 279642, 279641, 279632, 279630, 279559, 279550, 279491, 279488, 279487, 279483, 279441, 279439, 279435, 279410, 279391, 279355, 279354, 279349, 279348, 279344, 279342, 279312, 279310, 279309, 279274, 279273, 279270, 279268, 279267, 279265, 279264, 279242, 279238, 279235, 279234, 279232, 279208, 279207, 279201, 279199, 279157, 279155, 279130, 279123, 279122, 279118, 279117, 279107, 279106, 279075, 279074, 279073, 279069, 279068, 279044, 279043, 279041, 279036, 279035, 279008, 279007, 279005, 279000, 278999, 278964, 278963, 278960, 278959, 278941, 278939, 278936, 278915, 278914\\

 \textbf{PHOS:}\\
279234, 279235, 279238, 279242, 279264, 279265, 279267, 279268, 279270, 279273, 279274, 279309, 279310, 279312, 279342, 279348, 279349, 279354, 279355, 279391, 279435, 279439, 279441, 279483, 279487, 279488, 279491, 279550, 279559, 279630, 279632, 279641, 279642, 279676, 279679, 279682, 279683, 279684, 279687, 279688, 279689, 279715, 279718, 279719, 279747, 279749, 279773, 279826, 279827, 279830, 279853, 279855, 279879, 279880, 280051, 280052, 280066, 280107, 280108, 280111, 280114, 280118, 280126, 280131, 280134, 280135, 280140\\

 \subsection{LHC17o anchored to: LHC18a9, LHC18a9\_extra}

 \textbf{EDCtrigger:}\\
280282, 280283, 280284, 280285, 280286, 280348, 280349, 280350, 280351, 280352, 280374, 280375, 280403, 280412, 280413, 280415, 280419, 280518, 280519, 280546, 280547, 280550, 280551, 280574, 280576, 280581, 280583, 280613, 280634, 280636, 280637, 280647, 280648, 280650, 280671, 280673, 280676, 280679, 280705, 280706, 280729, 280753, 280754, 280755, 280756, 280757, 280761, 280762, 280763, 280764, 280765, 280766, 280767, 280768, 280786, 280787, 280793, 280842, 280844, 280845, 280847, 280848, 280849, 280854, 280856, 280880, 280881, 280897, 280936, 280943, 280947, 280994, 280996, 280997, 280998, 280999, 281032, 281033, 281035, 281036, 281060, 281061, 281062, 281080, 281081, 281179, 281180, 281181, 281189, 281190, 281191, 281212, 281213, 281240, 281241, 281242, 281243, 281244, 281271, 281273, 281275, 281277, 281301, 281321, 281415, 281443, 281444, 281446, 281449, 281450, 281563, 281568, 281569, 281574, 281580, 281705, 281706, 281707, 281713, 281750, 281751, 281753, 281892, 281893, 281894, 281895, 281915, 281916, 281918, 281920, 281928, 281931, 281932, 281939, 281961\\

 \textbf{EDC:}\\
280282, 280283, 280284, 280285, 280286, 280348, 280349, 280350, 280351, 280352, 280374, 280375, 280403, 280412, 280413, 280415, 280419, 280499, 280518, 280519, 280546, 280547, 280550, 280551, 280574, 280575, 280576, 280581, 280583, 280613, 280634, 280636, 280637, 280647, 280648, 280650, 280671, 280673, 280676, 280679, 280705, 280706, 280729, 280753, 280754, 280755, 280756, 280757, 280761, 280762, 280763, 280764, 280765, 280766, 280767, 280768, 280786, 280787, 280792, 280793, 280842, 280844, 280845, 280847, 280848, 280849, 280854, 280856, 280880, 280881, 280897, 280936, 280943, 280947, 280994, 280996, 280997, 280998, 280999, 281032, 281033, 281035, 281036, 281060, 281061, 281062, 281080, 281081, 281179, 281180, 281181, 281189, 281190, 281191, 281212, 281213, 281240, 281241, 281242, 281243, 281244, 281271, 281273, 281275, 281277, 281301, 281321, 281415, 281441, 281443, 281444, 281446, 281449, 281450, 281475, 281557, 281563, 281568, 281569, 281574, 281580, 281581, 281583, 281592, 281633, 281705, 281706, 281707, 281713, 281750, 281751, 281753, 281754, 281892, 281893, 281894, 281895, 281915, 281916, 281918, 281920, 281928, 281931, 281932, 281939, 281940, 281961\\

 \textbf{PCMEDC:}\\
280282, 280283, 280284, 280285, 280286, 280348, 280349, 280350, 280351, 280374, 280375, 280403, 280412, 280413, 280415, 280419, 280499, 280518, 280519, 280546, 280547, 280550, 280551, 280574, 280575, 280576, 280581, 280583, 280613, 280634, 280636, 280637, 280647, 280671, 280679, 280705, 280706, 280729, 280753, 280754, 280755, 280756, 280757, 280761, 280762, 280763, 280764, 280765, 280766, 280767, 280768, 280786, 280787, 280792, 280793, 280842, 280844, 280845, 280847, 280848, 280849, 280854, 280856, 280880, 280881, 280897, 280936, 280943, 280947, 280994, 280996, 280997, 280998, 280999, 281032, 281033, 281035, 281036, 281060, 281061, 281062, 281080, 281081, 281179, 281180, 281181, 281189, 281190, 281191, 281212, 281213, 281240, 281241, 281242, 281243, 281244, 281271, 281273, 281275, 281277, 281301, 281321, 281415, 281441, 281443, 281444, 281446, 281449, 281450, 281475, 281557, 281563, 281568, 281569, 281574, 281583, 281592, 281633, 281892, 281893, 281894, 281895, 281915, 281916, 281918, 281920, 281928, 281931, 281932, 281939, 281940, 281961\\

 \textbf{PCM:}\\
280282, 280283, 280284, 280285, 280286, 280290, 280310, 280312, 280348, 280349, 280350, 280351, 280374, 280375, 280403, 280405, 280406, 280412, 280413, 280415, 280419, 280443, 280445, 280446, 280447, 280448, 280490, 280499, 280518, 280519, 280546, 280547, 280550, 280551, 280574, 280575, 280576, 280581, 280583, 280613, 280634, 280636, 280637, 280639, 280645, 280647, 280671, 280679, 280681, 280705, 280706, 280729, 280753, 280754, 280755, 280756, 280757, 280761, 280762, 280763, 280764, 280765, 280766, 280767, 280768, 280786, 280787, 280792, 280793, 280842, 280844, 280845, 280847, 280848, 280849, 280854, 280856, 280880, 280881, 280890, 280897, 280936, 280940, 280943, 280947, 280990, 280994, 280996, 280997, 280998, 280999, 281032, 281033, 281035, 281036, 281060, 281061, 281062, 281080, 281081, 281179, 281180, 281181, 281189, 281190, 281191, 281212, 281213, 281240, 281241, 281242, 281243, 281244, 281271, 281273, 281275, 281277, 281301, 281321, 281415, 281441, 281443, 281444, 281446, 281449, 281450, 281475, 281477, 281509, 281511, 281557, 281562, 281563, 281568, 281569, 281574, 281583, 281592, 281633, 281892, 281893, 281894, 281895, 281915, 281916, 281918, 281920, 281928, 281931, 281932, 281939, 281940, 281953, 281956, 281961\\

 \textbf{PHOS:}\\
280282, 280283, 280284, 280285, 280286, 280290, 280310, 280312, 280348, 280349, 280350, 280351, 280352, 280374, 280375, 280403, 280405, 280406, 280412, 280413, 280415, 280419, 280443, 280445, 280446, 280447, 280448, 280490, 280499, 280518, 280519, 280546, 280547, 280551, 280574, 280575, 280576, 280581, 280583, 280613, 280634, 280636, 280637, 280639, 280645, 280647, 280648, 280650, 280671, 280673, 280676, 280679, 280681, 280705, 280729, 280753, 280755, 280756, 280757, 280761, 280762, 280763, 280764, 280765, 280766, 280767, 280768, 280786, 280787, 280792, 280793, 280842, 280844, 280845, 280847, 280848, 280849, 280854, 280856, 280880, 280881, 280890, 280897, 280936, 280940, 280943, 280947, 280990, 280994, 280996, 280997, 280998, 280999, 281032, 281033, 281035, 281036, 281060, 281061, 281062, 281080, 281081, 281179, 281180, 281181, 281189, 281190, 281191, 281212, 281213, 281240, 281241, 281242, 281243, 281244, 281271, 281273, 281275, 281277, 281301, 281321, 281415, 281441, 281443, 281444, 281446, 281449, 281450, 281475, 281477, 281509, 281557, 281562, 281563, 281592, 281633, 281705, 281706, 281707, 281709, 281713, 281741, 281750, 281751, 281753, 281754, 281755, 281756, 281892, 281893, 281894, 281895, 281915, 281916, 281918, 281931, 281932, 281939, 281940, 281953, 281956, 281961\\

 \subsection{LHC17r anchored to: LHC18a1, LHC18a1\_extra}

 \textbf{EDCtrigger:}\\
282544, 282545, 282546, 282579, 282580, 282606, 282607, 282608, 282609, 282620, 282622, 282651, 282666, 282667, 282670, 282671, 282673, 282676, 282677, 282700, 282702, 282703, 282704\\

 \textbf{EDC:}\\
282544, 282545, 282546, 282573, 282579, 282580, 282606, 282607, 282608, 282609, 282618, 282620, 282622, 282629, 282651, 282666, 282667, 282670, 282671, 282673, 282676, 282677, 282700, 282702, 282703, 282704\\

 \textbf{PCMEDC:}\\
 282544, 282545, 282546, 282573, 282579, 282580, 282606, 282607, 282608, 282609, 282618, 282620, 282622, 282629, 282651, 282666, 282667, 282670, 282671, 282673, 282676, 282677, 282700, 282702, 282703, 282704\\

 \textbf{PCM:}\\
282528, 282544, 282545, 282546, 282573, 282575, 282579, 282580, 282606, 282607, 282608, 282609, 282618, 282620, 282622, 282629, 282651, 282653, 282666, 282667, 282668, 282670, 282671, 282673, 282676, 282677, 282700, 282702, 282703, 282704\\

 \textbf{PHOS:}\\
282704, 282703, 282702, 282700, 282677, 282676, 282671, 282670, 282668, 282667, 282666, 282653, 282651, 282629, 282620, 282609, 282608, 282607, 282606, 282580, 282579, 282575, 282573, 282546, 282545, 282544, 282528\\


 \subsection{LHC18b anchored to:  }

  \textbf{EDCtrigger:}\\
 285009, 285011, 285012, 285013, 285014, 285015, 285106, 285108, 285125, 285127, 285165, 285200, 285202, 285203, 285222, 285224, 285327, 285328, 285347, 285364, 285365, 285396 \\

 \textbf{EDC:}\\
 285009, 285011, 285012, 285013, 285014, 285015, 285106, 285108, 285125, 285127, 285165, 285200, 285202, 285203, 285222, 285224, 285327, 285328, 285347, 285364, 285365, 285396 \\

 \textbf{PCMEDC:}\\
 285009, 285011, 285012, 285013, 285014, 285015, 285106, 285108, 285125, 285127, 285165, 285200, 285202, 285203, 285222, 285224, 285327, 285328, 285347, 285364, 285365, 285396  \\

 \textbf{PHOS:}\\
285008, 285009, 285010, 285011, 285012, 285013, 285014, 285015, 285064, 285065, 285066, 285106, 285108, 285125, 285127, 285165, 285200, 285202, 285203, 285222, 285224, 285327, 285328, 285347, 285364, 285365, 285396, 285447\\

 \subsection{LHC18d anchored to:  }

 \textbf{EDCtrigger:}\\
 286014, 286025, 286064, 286124, 286127, 286129, 286130, 286159, 286198, 286201, 286202, 286203, 286229, 286230, 286231, 286254, 286255, 286257, 286258, 286261, 286263, 286282, 286284, 286287, 286288, 286289, 286308, 286309, 286310, 286311, 286313, 286314, 286336, 286337, 286340, 286341, 286345, 286348, 286349, 286350\\

 \textbf{EDC:}\\
 286014, 286025, 286064, 286124, 286127, 286129, 286130, 286159, 286198, 286201, 286202, 286203, 286229, 286230, 286231, 286254, 286255, 286257, 286258, 286261, 286263, 286282, 286284, 286287, 286288, 286289, 286308, 286309, 286310, 286311, 286313, 286314, 286336, 286337, 286340, 286341, 286345, 286348, 286349, 286350, 286348, 286349, 286350 \\


 \textbf{PCMEDC:}\\
 286014, 286025, 286064, 286124, 286127, 286129, 286130, 286159, 286198, 286201, 286202, 286203, 286229, 286230, 286231, 286254, 286255, 286257, 286258, 286261, 286263, 286282, 286284, 286287, 286288, 286289, 286308, 286309, 286310, 286311, 286313, 286314, 286336, 286337, 286340, 286341, 286345, 286348, 286349, 286350 \\

 \textbf{PHOS:}\\
285978, 285979, 285980, 286014, 286025, 286027, 286028, 286030, 286064, 286124, 286127, 286129, 286130, 286159, 286198, 286199, 286201, 286202, 286203, 286229, 286230, 286231, 286254, 286255, 286257, 286258, 286261, 286263, 286282, 286284, 286287, 286288, 286289, 286308, 286309, 286310, 286311, 286312, 286313, 286314, 286336, 286337, 286340, 286341, 286345, 286348, 286349, 286350\\

 \subsection{LHC18e anchored to:  }

  \textbf{EDCtrigger:}\\
 286380, 286426, 286428, 286454, 286455, 286482, 286501, 286502, 286508, 286509, 286511, 286566, 286567, 286568, 286569, 286591, 286592, 286653, 286661, 286695, 286731, 286799, 286801, 286846, 286848, 286850, 286852, 286874, 286876, 286907, 286908, 286910, 286911, 286930, 286931, 286932, 286936, 286937 \\

 \textbf{EDC:}\\
 286380, 286426, 286427, 286428, 286454, 286455, 286482, 286501, 286502, 286508, 286509, 286511, 286566, 286567, 286568, 286569, 286591, 286592, 286594, 286653, 286661, 286695, 286731, 286799, 286801, 286846, 286848, 286850, 286852, 286874, 286876, 286877, 286907, 286908, 286910, 286911, 286930, 286931, 286932, 286936, 286937 \\


 \textbf{PCMEDC:}\\
 286380, 286426, 286428, 286454, 286455, 286482, 286501, 286502, 286508, 286509, 286511, 286566, 286567, 286568, 286569, 286591, 286592, 286653, 286661, 286695, 286731, 286799, 286801, 286846, 286848, 286850, 286852, 286874, 286876, 286907, 286908, 286910, 286911, 286930, 286931, 286932, 286936, 286937 \\

 \textbf{PHOS:}\\
286380, 286426, 286427, 286428, 286454, 286455, 286482, 286501, 286502, 286508, 286509, 286511, 286566, 286567, 286568, 286569, 286591, 286592, 286594, 286633, 286653, 286661, 286695, 286731, 286799, 286801, 286805, 286809, 286810, 286846, 286848, 286850, 286852, 286874, 286876, 286877, 286907, 286908, 286910, 286911, 286930, 286931, 286932, 286933, 286936, 286937\\

 \subsection{LHC18f anchored to:  }

	\textbf{EDCtrigger:}\\
	287000, 287021, 287063, 287064, 287071, 287077, 287137, 287155, 287185, 287201, 287209, 287248, 287249, 287250, 287254, 287283, 287323, 287324, 287325, 287343, 287346, 287347, 287349, 287353, 287355, 287356, 287360, 287380, 287381, 287385, 287387, 287388, 287389, 287413, 287480, 287481, 287484, 287486, 287513, 287516, 287517, 287520, 287521, 287573, 287575, 287576, 287578, 287657, 287783, 287876, 287877, 287883, 287884, 287885, 287911, 287912, 287913, 287915, 287923, 287941, 287975, 287977 \\

	\textbf{EDC:}\\
	287000, 287021, 287063, 287064, 287071, 287072, 287077, 287137, 287155, 287185, 287201, 287208, 287209, 287248, 287249, 287250, 287254, 287283, 287323, 287324, 287325, 287343, 287346, 287347, 287349, 287353, 287355, 287356, 287360, 287380, 287381, 287385, 287387, 287388, 287389, 287413, 287480, 287481, 287484, 287486, 287513, 287516, 287517, 287518, 287520, 287521, 287524, 287573, 287575, 287576, 287578, 287657, 287658, 287783, 287784, 287876, 287877, 287883, 287884, 287885, 287911, 287912, 287913, 287915, 287923, 287941, 287975, 287977, 287923, 287941, 287975, 287977 \\

	\textbf{PCMEDC:}\\
	287000, 287021, 287063, 287064, 287071, 287077, 287137, 287155, 287185, 287201, 287209, 287248, 287249, 287250, 287254, 287283, 287323, 287324, 287325, 287343, 287346, 287347, 287349, 287353, 287355, 287356, 287360, 287380, 287381, 287385, 287387, 287388, 287389, 287413, 287480, 287481, 287484, 287486, 287513, 287516, 287517, 287520, 287521, 287573, 287575, 287576, 287578, 287657, 287783, 287876, 287877, 287883, 287884, 287885, 287911, 287912, 287913, 287915, 287923, 287941, 287975, 287977  \\

 \textbf{PHOS:}\\
287000, 287021, 287063, 287064, 287066, 287071, 287072, 287077, 287137, 287155, 287185, 287201, 287202, 287203, 287204, 287208, 287209, 287248, 287249, 287250, 287251, 287254, 287283, 287323, 287324, 287325, 287343, 287344, 287346, 287347, 287349, 287353, 287355, 287356, 287360, 287380, 287381, 287385, 287387, 287388, 287389, 287413, 287451, 287480, 287481, 287484, 287486, 287513, 287516, 287517, 287518, 287520, 287521, 287524, 287573, 287575, 287576, 287578, 287616, 287654, 287656, 287657, 287658, 287783, 287784, 287876, 287877, 287883, 287884, 287885, 287911, 287912, 287913, 287915, 287923, 287941, 287975, 287977\\

 \subsection{LHC18g anchored to:  }

 \textbf{EDCtrigger:}\\
 288619, 288650 \\

 \textbf{EDC:}\\
 288619, 288650 \\

 \textbf{PCMEDC:}\\
 288619, 288650 \\

 \textbf{PHOS:}\\
288619, 288640, 288642, 288644, 288650, 288687, 288689, 288690, 288743, 288748, 288750\\

 \subsection{LHC18h anchored to:  }

 \textbf{EDCtrigger:}\\
 288804, 288806\\

 \textbf{EDC:}\\
 288804, 288806\\

 \textbf{PCMEDC:}\\
288804, 288806\\

 \textbf{PHOS:}\\
288804, 288806\\

 \subsection{LHC18i anchored to:  }

 \textbf{EDCtrigger:}\\
  288861, 288862, 288864, 288868, 288897, 288903, 288908, 288909\\

 \textbf{EDC:}\\
 288861, 288862, 288864, 288868, 288897, 288903, 288908, 288909\\

 \textbf{PCMEDC:}\\
288861, 288862, 288864, 288868, 288897, 288903, 288908, 288909 \\

 \textbf{PHOS:}\\
288861, 288862, 288863, 288864, 288868, 288897, 288902, 288903, 288908, 288909\\

 \subsection{LHC18j anchored to:  }

 \textbf{EDCtrigger:}\\
 288943\\

 \textbf{EDC:}\\
 288943\\

 \textbf{PCMEDC:}\\
288943\\

 \textbf{PHOS:}\\
288943\\

 \subsection{LHC18k anchored to:  }

 \textbf{EDCtrigger:}\\
 289165, 289166, 289169, 289172, 289176, 289177, 289198, 289199, 289200, 289201\\

 \textbf{EDC:}\\
 289165, 289166, 289169, 289172, 289176, 289177, 289198, 289199, 289200, 289201\\

 \textbf{PCMEDC:}\\
289165, 289166, 289169, 289172, 289176, 289177, 289198, 289199, 289200, 289201  \\

 \textbf{PHOS:}\\
289165, 289166, 289167, 289169, 289172, 289175, 289176, 289177, 289198, 289199, 289200, 289201\\

 \subsection{LHC18l anchored to:  }

 \textbf{EDCtrigger:}\\
 289240, 289242, 289243, 289253, 289275, 289276, 289277, 289278, 289280, 289281, 289300, 289303, 289306, 289308, 289309, 289354, 289355, 289356, 289365, 289366, 289367, 289368, 289369, 289370, 289373, 289374, 289426, 289444, 289462, 289463, 289465, 289466, 289468, 289493, 289494, 289521, 289547, 289721, 289729, 289731, 289732, 289757, 289775, 289808, 289811, 289815, 289816, 289817, 289818, 289830, 289849, 289854, 289855, 289856, 289857, 289879, 289880, 289884, 289928, 289935, 289940, 289943, 289965, 289966, 289971 \\

 \textbf{EDC:}\\
 289240, 289242, 289243, 289253, 289275, 289276, 289277, 289278, 289280, 289281, 289300, 289303, 289306, 289308, 289309, 289354, 289355, 289356, 289365, 289366, 289367, 289368, 289369, 289370, 289373, 289374, 289426, 289444, 289462, 289463, 289465, 289466, 289468, 289493, 289494, 289521, 289547, 289721, 289729, 289731, 289732, 289757, 289775, 289808, 289811, 289815, 289816, 289817, 289818, 289830, 289849, 289854, 289855, 289856, 289857, 289879, 289880, 289884, 289928, 289935, 289940, 289943, 289965, 289966, 289971, 289241, 289254, 289353, 289724, 289814, 289966\\

 \textbf{PCMEDC:}\\
 289240, 289242, 289243, 289253, 289275, 289276, 289277, 289278, 289280, 289281, 289300, 289303, 289306, 289308, 289309, 289354, 289355, 289356, 289365, 289366, 289367, 289368, 289369, 289370, 289373, 289374, 289426, 289444, 289462, 289463, 289465, 289466, 289468, 289493, 289494, 289521, 289547, 289721, 289729, 289731, 289732, 289757, 289775, 289808, 289811, 289815, 289816, 289817, 289818, 289830, 289849, 289854, 289855, 289856, 289857, 289879, 289880, 289884, 289928, 289935, 289940, 289943, 289965, 289966, 289971  \\

 \textbf{PHOS:}\\
289240, 289241, 289242, 289243, 289247, 289249, 289253, 289254, 289275, 289276, 289277, 289278, 289280, 289281, 289300, 289303, 289306, 289308, 289309, 289330, 289331, 289353, 289354, 289355, 289356, 289363, 289365, 289366, 289367, 289368, 289369, 289370, 289373, 289374, 289426, 289444, 289462, 289463, 289465, 289466, 289468, 289493, 289494, 289521, 289547, 289574, 289576, 289577, 289582, 289625, 289632, 289634, 289657, 289658, 289659, 289660, 289664, 289666, 289721, 289723, 289724, 289729, 289731, 289732, 289757, 289775, 289808, 289811, 289814, 289815, 289816, 289817, 289818, 289830, 289849, 289852, 289854, 289855, 289856, 289857, 289879, 289880, 289884, 289928, 289931, 289935, 289940, 289941, 289943, 289965, 289966, 289971\\

 \subsection{LHC18m anchored to:  }
\textbf{EDCtrigger:}\\
290323, 290324, 290327, 290350, 290374, 290375, 290399, 290401, 290411, 290412, 290425, 290426, 290427, 290428, 290456, 290458, 290459, 290469, 290499, 290500, 290538, 290539, 290540, 290544, 290549, 290550, 290553, 290588, 290590, 290612, 290613, 290614, 290615, 290627, 290632, 290645, 290658, 290660, 290665, 290941, 290943, 290944, 290948, 290974, 290975, 290976, 290979, 290980, 291002, 291003, 291004, 291005, 291035, 291037, 291041, 291065, 291066, 291069, 291093, 291100, 291110, 291111, 291116, 291143, 291188, 291209, 291240, 291257, 291262, 291263, 291265, 291266, 291282, 291283, 291284, 291285, 291286, 291360, 291361, 291362, 291373, 291375, 291377, 291397, 291399, 291402, 291417, 291419, 291420, 291424, 291446, 291447, 291453, 291456, 291457, 291481, 291482, 291484, 291485, 292242, 292397, 292398, 292405, 292406, 292428, 292429, 292430, 292432, 292434, 292456, 292457, 292460, 292461, 292495, 292496, 292497, 292500, 292521, 292523, 292524, 292526, 292553, 292554, 292557, 292559, 292560, 292563, 292584, 292586, 292693, 292695, 292696, 292698, 292701, 292704, 292737, 292739, 292747, 292748, 292750, 292809, 292810, 292811, 292831, 292832, 292836\\

\textbf{EDC:}\\
290323, 290324, 290327, 290350, 290374, 290375, 290376, 290399, 290401, 290411, 290412, 290425, 290426, 290427, 290428, 290456, 290458, 290459, 290469, 290499, 290500, 290538, 290539, 290540, 290544, 290549, 290550, 290553, 290588, 290590, 290612, 290613, 290614, 290615, 290627, 290632, 290645, 290658, 290660, 290665, 290687, 290689, 290692, 290696, 290699, 290721, 290742, 290764, 290766, 290769, 290774, 290787, 290790, 290841, 290843, 290846, 290848, 290860, 290862, 290886, 290887, 290892, 290894, 290895, 290932, 290935, 290941, 290943, 290944, 290948, 290974, 290975, 290976, 290979, 290980, 291002, 291003, 291004, 291005, 291035, 291037, 291041, 291065, 291066, 291069, 291093, 291100, 291101, 291110, 291111, 291116, 291143, 291188, 291209, 291240, 291257, 291262, 291263, 291265, 291266, 291282, 291283, 291284, 291285, 291286, 291360, 291361, 291362, 291373, 291375, 291377, 291397, 291399, 291402, 291416, 291417, 291419, 291420, 291424, 291446, 291447, 291453, 291456, 291457, 291481, 291482, 291484, 291485, 291590, 291614, 291615, 291618, 291622, 291624, 291626, 291657, 291661, 291665, 291690, 291692, 291694, 291697, 291698, 291702, 291706, 291729, 291755, 291756, 291760, 291768, 291769, 291795, 291796, 291803, 291942, 291943, 291944, 291945, 291946, 291948, 291953, 291976, 291977, 291982, 292012, 292040, 292060, 292061, 292062, 292067, 292075, 292080, 292081, 292106, 292107, 292108, 292109, 292114, 292115, 292140, 292160, 292161, 292162, 292163, 292164, 292166, 292167, 292168, 292192, 292218, 292240, 292242, 292265, 292270, 292273, 292274, 292298, 292397, 292398, 292405, 292406, 292428, 292429, 292430, 292432, 292434, 292456, 292457, 292460, 292461, 292495, 292496, 292497, 292500, 292521, 292523, 292524, 292526, 292553, 292554, 292557, 292559, 292560, 292563, 292584, 292586, 292693, 292695, 292696, 292698, 292701, 292704, 292737, 292739, 292747, 292748, 292750, 292803, 292804, 292809, 292810, 292811, 292831, 292832, 292836, 292839, 292832, 292836, 292839 \\

\textbf{PCMEDC:}\\
290323, 290324, 290327, 290350, 290374, 290375, 290376, 290399, 290401, 290411, 290412, 290425, 290426, 290427, 290428, 290456, 290458, 290459, 290469, 290499, 290500, 290538, 290539, 290540, 290544, 290549, 290550, 290553, 290588, 290590, 290612, 290613, 290614, 290615, 290627, 290632, 290645, 290658, 290660, 290665, 290687, 290689, 290692, 290696, 290699, 290721, 290764, 290766, 290769, 290774, 290787, 290790, 290841, 290843, 290846, 290848, 290860, 290862, 290886, 290887, 290892, 290894, 290895, 290932, 290935, 290941, 290943, 290944, 290948, 290974, 290975, 290976, 290979, 290980, 291002, 291003, 291004, 291005, 291035, 291037, 291041, 291065, 291066, 291069, 291093, 291100, 291110, 291111, 291116, 291143, 291188, 291209, 291240, 291257, 291262, 291263, 291265, 291266, 291282, 291283, 291284, 291285, 291286, 291360, 291361, 291362, 291373, 291375, 291377, 291397, 291399, 291402, 291417, 291419, 291420, 291424, 291446, 291447, 291453, 291456, 291457, 291481, 291482, 291484, 291485, 291614, 291615, 291618, 291622, 291624, 291626, 291657, 291661, 291665, 291690, 291692, 291694, 291698, 291706, 291729, 291755, 291756, 291760, 291768, 291795, 291796, 291803, 291942, 291943, 291944, 291945, 291946, 291948, 291953, 291976, 291977, 291982, 292012, 292040, 292060, 292061, 292062, 292067, 292075, 292080, 292081, 292106, 292107, 292108, 292109, 292114, 292115, 292140, 292160, 292161, 292162, 292163, 292164, 292166, 292167, 292168, 292192, 292218, 292240, 292242, 292265, 292270, 292273, 292274, 292298, 292397, 292398, 292405, 292406, 292428, 292429, 292430, 292432, 292434, 292456, 292457, 292460, 292461, 292495, 292496, 292497, 292500, 292521, 292523, 292524, 292526, 292553, 292554, 292557, 292559, 292560, 292563, 292584, 292586, 292693, 292695, 292696, 292698, 292701, 292704, 292737, 292739, 292747, 292748, 292750, 292809, 292810, 292811, 292831, 292832, 292836 \\

 \subsection{LHC18n anchored to:  }

 \textbf{EDCtrigger:}\\
 293357, 293359\\

 \textbf{EDC:}\\
 293357, 293359\\

 \textbf{PCMEDC:}\\
293357, 293359\\

 \textbf{PHOS:}\\
293357, 293359\\

 \subsection{LHC18o anchored to:  }

 \textbf{EDCtrigger:}\\
 293368, 293386, 293392, 293413, 293424, 293474, 293475, 293494, 293496, 293497, 293570, 293571, 293582, 293587, 293588, 293686, 293690, 293691, 293695, 293696, 293740, 293741, 293770, 293773, 293774, 293776, 293799, 293805, 293806, 293809, 293829, 293830, 293831, 293856, 293886, 293891, 293893, 293896, 293898\\

 \textbf{EDC:}\\
 293368, 293386, 293392, 293413, 293424, 293474, 293475, 293494, 293496, 293497, 293570, 293571, 293578, 293582, 293587, 293588, 293686, 293689, 293690, 293691, 293695, 293696, 293740, 293741, 293770, 293773, 293774, 293776, 293799, 293805, 293806, 293809, 293829, 293830, 293831, 293856, 293886, 293891, 293893, 293896, 293898\\

 \textbf{PCMEDC:}\\
 293368, 293386, 293392, 293413, 293424, 293474, 293475, 293494, 293496, 293497, 293570, 293571, 293582, 293587, 293588, 293686, 293690, 293691, 293695, 293696, 293740, 293741, 293770, 293773, 293774, 293776, 293799, 293805, 293806, 293809, 293829, 293830, 293831, 293856, 293886, 293891, 293893, 293896, 293898 \\

 \textbf{PHOS:}\\
293368, 293386, 293392, 293413, 293424, 293474, 293475, 293494, 293496, 293497, 293570, 293571, 293573, 293578, 293579, 293582, 293583, 293587, 293588, 293686, 293689, 293690, 293691, 293692, 293695, 293696, 293698, 293740, 293741, 293770, 293773, 293774, 293776, 293799, 293802, 293805, 293806, 293807, 293809, 293829, 293830, 293831, 293856, 293886, 293891, 293893, 293896, 293898\\

 \subsection{LHC18p anchored to:  }

 \textbf{EDCtrigger:}\\
 294009, 294010, 294011, 294012, 294013, 294131, 294152, 294154, 294155, 294156, 294199, 294200, 294208, 294210, 294212, 294241, 294242, 294305, 294307, 294310, 294524, 294525, 294526, 294529, 294530, 294531, 294553, 294556, 294558, 294562, 294563, 294586, 294587, 294588, 294590, 294591, 294593, 294632, 294633, 294634, 294636, 294653, 294715, 294716, 294721, 294722, 294741, 294742, 294743, 294744, 294745, 294746, 294747, 294769, 294774, 294775, 294805, 294815, 294816, 294817, 294818, 294852, 294875, 294877, 294883, 294884, 294925 \\

 \textbf{EDC:}\\
 294009, 294010, 294011, 294012, 294013, 294131, 294152, 294154, 294155, 294156, 294199, 294200, 294208, 294210, 294212, 294241, 294242, 294305, 294307, 294310, 294524, 294525, 294526, 294529, 294530, 294531, 294553, 294556, 294558, 294562, 294563, 294586, 294587, 294588, 294590, 294591, 294593, 294632, 294633, 294634, 294636, 294653, 294715, 294716, 294721, 294722, 294741, 294742, 294743, 294744, 294745, 294746, 294747, 294769, 294774, 294775, 294805, 294815, 294816, 294817, 294818, 294852, 294875, 294877, 294883, 294884, 294925\\

 \textbf{PCMEDC:}\\
294009, 294010, 294011, 294012, 294013, 294131, 294152, 294154, 294155, 294156, 294199, 294200, 294208, 294210, 294212, 294241, 294242, 294305, 294307, 294310, 294524, 294525, 294526, 294529, 294530, 294531, 294553, 294556, 294558, 294562, 294563, 294586, 294587, 294588, 294590, 294591, 294593, 294632, 294633, 294634, 294636, 294653, 294715, 294716, 294721, 294722, 294741, 294742, 294743, 294744, 294745, 294746, 294747, 294769, 294774, 294775, 294805, 294815, 294816, 294817, 294818, 294852, 294875, 294877, 294883, 294884, 294925\\

 \textbf{PHOS:}\\
294009, 294010, 294011, 294012, 294013, 294128, 294131, 294152, 294154, 294155, 294156, 294199, 294200, 294201, 294205, 294208, 294210, 294212, 294241, 294242, 294305, 294307, 294308, 294310, 294502, 294503, 294524, 294525, 294526, 294527, 294529, 294530, 294531, 294553, 294556, 294558, 294562, 294563, 294586, 294587, 294588, 294590, 294591, 294593, 294620, 294632, 294633, 294634, 294636, 294653, 294703, 294710, 294715, 294716, 294718, 294721, 294722, 294741, 294742, 294743, 294744, 294745, 294746, 294747, 294749, 294769, 294772, 294774, 294775, 294805, 294809, 294813, 294815, 294816, 294817, 294818, 294852, 294875, 294877, 294880, 294883, 294884, 294916, 294925\\

%
%

\subsection{LHC16x, MB INT7}
\label{subsec:goodRuns16}

Using that selection of runs, the quality assurance has been performed as it is described in the respective chapters that can be found within this note. Although all different observables (see \hyperref[tab:trackSelection]{Table~\ref*{tab:trackSelection}},  \hyperref[tab:PID]{Table~\ref*{tab:PID}}, \hyperref[tab:PhotonCuts]{Table~\ref*{tab:PhotonCuts}} and \hyperref[tab:clusterSelection]{Table~\ref*{tab:clusterSelection}}) have been checked in great detail for their periodwise and runwise agreement, the most important observables to distinguish ``good`` and ``bad`` runs are: \vspace{-0.2cm}
\begin{itemize}
\item[-] Number of $\pi^{0}$ candidates per event, $\pi^{0}$ mass and width \vspace{-0.3cm}
\item[-] Conversion Photon Candidates (short: conversions) per event, \acs{EMCal} Photon Candidates (short: clusters) per event \vspace{-0.3cm}
\item[-] Energy distributions of conversions, clusters \vspace{-0.3cm}
\item[-] $\eta$/$\phi$ distributions of selected conversions, clusters \vspace{-0.3cm}
\item[-] ``Event properties `` like number of charged tracks, z-vertex distributions as well as fraction of events discarded due to pileup or missing primary vertices \vspace{-0.2cm}
\end{itemize}
If a run deviates from the global mean value of the respective observable for more than $2\sigma$, it is considered as ``bad``. This is being done unless there is no clear explanation for a different behavior like some modules of \acs{EMCal} being switched off, for example, which leads to a decrease in the number of good \acs{EMCal} photon candidates. Therefore, we always compared data and Monte Carlo as well to ensure that Monte Carlo describes the recorded data reasonably well. Also, if trending points in data and Monte Carlo do not show the same behavior, which means if values of observables change and if they are not properly followed by the other respective set, the run is also considered as ``bad`` unless the reason for the deviation cannot be clearly identified and fixed.\TODO{All}{Add Runlists}
%
% \begin{table}[h]
% \hspace*{-0.2cm}
% \small
% \centering
% \begin{tabular}{ll}
% \toprule
% \textbf{LHC16d - PCM Good Runs PCM INT7} \\ \midrule
%     252235, 252238, 252248, 252271, 252310, 252313, 252315, 252317, 252319, 252322, \\ \midrule
%     252325, 252326, 252330, 252332, 252336, 252368, 252370, 252371, 252374, 252375 \\
%     \bottomrule
%     \end{tabular}
%     \caption{Runs used in this analysis for LHC16x, minimum bias trigger INT7 in PCM. The set of runs is identical for data and corresponding Monte Carlo productions that are used.}
%     \label{tab:runs16d}
% \end{table}
%
% \begin{table}[h]
% \hspace*{-0.2cm}
% \small
% \centering
% \begin{tabular}{ll}
% \toprule
% \textbf{LHC16e - PCM Good Runs PCM INT7} \\ \midrule
%     252858, 252867, 253437, 253478, 253481, 253482, 253488, 253517, 253529, 253530, \\ %\midrule
%     253563, 253589, 253591 \\
%     \bottomrule
%     \end{tabular}
%     \caption{Runs used in this analysis for LHC16x, minimum bias trigger INT7 in PCM. The set of runs is identical for data and corresponding Monte Carlo productions that are used.}
%     \label{tab:runs16e}
% \end{table}

% 	\begin{table}[h]
% 	\hspace*{-0.2cm}
% 	\small
% 	\centering
% 	\begin{tabular}{ll}
% 	\toprule
% 	\textbf{LHC16g - PCM Good Runs PCM INT7} \\ \midrule
% 		254128, 254147, 254148, 254149, 254174, 254175, 254178, 254193, 254199, 254204, \\ \midrule
% 		254205, 254293, 254302, 254303, 254304, 254330, 254331, 254332 \\
% 		\bottomrule
% 		\end{tabular}
% 		\caption{Runs used in this analysis for LHC16x, minimum bias trigger INT7 in PCM. The set of runs is identical for data and corresponding Monte Carlo productions that are used.}
% 		\label{tab:runs16g}
% 	\end{table}
%
% 	\begin{table}[h]
% 	\hspace*{-0.2cm}
% 	\small
% 	\centering
% 	\begin{tabular}{ll}
% 	\toprule
% 	\textbf{LHC16h - PCM Good Runs PCM INT7} \\ \midrule
% 		254418, 254419, 254422, 254604, 254606, 254608, 254621, 254629, 254630, 254632, \\ \midrule
% 		254640, 254644, 254646, 254648, 254649, 254651, 254652, 254653, 254654, 255079, \\ \midrule
% 		255082, 255085, 255086, 255091, 255111, 255154, 255159, 255162, 255167, 255171, \\ \midrule
% 		255173, 255174, 255176, 255177, 255240, 255242, 255247, 255248, 255249, 255251, \\ \midrule
% 		255252, 255253, 255255, 255256, 255275, 255276, 255280, 255283, 255350, 255351, \\ \midrule
% 		255352, 255398, 255402, 255407, 255415, 255418, 255419, 255420, 255421, 255440, \\ \midrule
% 		255442, 255447, 255463, 255465, 255466, 255467 \\
% 		\bottomrule
% 		\end{tabular}
% 		\caption{Runs used in this analysis for LHC16x, minimum bias trigger INT7 in PCM. The set of runs is identical for data and corresponding Monte Carlo productions that are used.}
% 		\label{tab:runs16h}
% 	\end{table}
%
% 	\begin{table}[h]
% 	\hspace*{-0.2cm}
% 	\small
% 	\centering
% 	\begin{tabular}{ll}
% 	\toprule
% 	\textbf{LHC16i - PCM Good Runs PCM INT7} \\ \midrule
% 		255539, 255540, 255541, 255542, 255543, 255577, 255582, 255583, 255591, 255614, \\ \midrule
% 		255615, 255616, 255617, 255618 \\
% 		\bottomrule
% 		\end{tabular}
% 		\caption{Runs used in this analysis for LHC16x, minimum bias trigger INT7 in PCM. The set of runs is identical for data and corresponding Monte Carlo productions that are used.}
% 		\label{tab:runs16i}
% 	\end{table}
%
% 	\begin{table}[h]
% 	\hspace*{-0.2cm}
% 	\small
% 	\centering
% 	\begin{tabular}{ll}
% 	\toprule
% 	\textbf{LHC16j - PCM Good Runs PCM INT7} \\ \midrule
% 		256204, 256207, 256210, 256212, 256213, 256215, 256219, 256222, 256223, 256227, \\ \midrule
% 		256228, 256231, 256281, 256282, 256283, 256284, 256287, 256289, 256290, 256292, \\ \midrule
% 		256295, 256297, 256298, 256299, 256302, 256307, 256309, 256311, 256356, 256357, \\ \midrule
% 		256361, 256362, 256363, 256364, 256365, 256366, 256368, 256371, 256372, 256373, \\ \midrule
% 		256415, 256417, 256418, 256420 \\
% 		\bottomrule
% 		\end{tabular}
% 		\caption{Runs used in this analysis for LHC16x, minimum bias trigger INT7 in PCM. The set of runs is identical for data and corresponding Monte Carlo productions that are used.}
% 		\label{tab:runs16j}
% 	\end{table}
%
% 	\begin{table}[h]
% 	\hspace*{-0.2cm}
% 	\small
% 	\centering
% 	\begin{tabular}{ll}
% 	\toprule
% 	\textbf{LHC16k - PCM Good Runs PCM INT7} \\ \midrule
% 		256504, 256506, 256510, 256512, 256552, 256554, 256556, 256557, 256560, 256561, \\ \midrule
% 		256562, 256564, 256565, 256567, 256589, 256619, 256620, 256658, 256676, 256677, \\ \midrule
% 		256681, 256684, 256691, 256692, 256694, 256695, 256697, 256941, 256942, 256944, \\ \midrule
% 		257012, 257021, 257026, 257077, 257080, 257082, 257083, 257084, 257086, 257092, \\ \midrule
% 		257095, 257100, 257136, 257137, 257138, 257139, 257140, 257141, 257142, 257144, \\ \midrule
% 		257145, 257204, 257206, 257209, 257224, 257260, 257318, 257320, 257322, 257330, \\ \midrule
% 		257358, 257364, 257433, 257457, 257468, 257474, 257487, 257488, 257490, 257491, \\ \midrule
% 		257492, 257530, 257531, 257537, 257539, 257540, 257541, 257560, 257561, 257562, \\ \midrule
% 		257566, 257587, 257588, 257590, 257592, 257594, 257595, 257604, 257605, 257606, \\ \midrule
% 		257630, 257632, 257635, 257636, 257642, 257644, 257682, 257684, 257685, 257687, \\ \midrule
% 		257688, 257689, 257691, 257692, 257694, 257697, 257724, 257725, 257727, 257733, \\ \midrule
% 		257734, 257735, 257737, 257754, 257757, 257765, 257773, 257797, 257798, 257799, \\ \midrule
% 		257800, 257803, 257804, 257850, 257851, 257853, 257855, 257893, 257901, 257912, \\ \midrule
% 		257932, 257936, 257937, 257939, 257957, 257958, 257960, 257963, 257979, 257986, \\ \midrule
% 		257989, 257992, 258003, 258008, 258012, 258014, 258017, 258019, 258039, 258041, \\ \midrule
% 		258042, 258045, 258048, 258049, 258053, 258059, 258060, 258062, 258063, 258107, \\ \midrule
% 		258108, 258109, 258113, 258114, 258117, 258178, 258197, 258198, 258202, 258203, \\ \midrule
% 		258204, 258256, 258257, 258258, 258270, 258271, 258273, 258274, 258278, 258299, \\ \midrule
% 		258301, 258302, 258303, 258306, 258307, 258332, 258336, 258359, 258391, 258393, \\ \midrule
% 		258426, 258452, 258454, 258456, 258477, 258499, 258537 \\
% 		\bottomrule
% 		\end{tabular}
% 		\caption{Runs used in this analysis for LHC16x, minimum bias trigger INT7 in PCM. The set of runs is identical for data and corresponding Monte Carlo productions that are used.}
% 		\label{tab:runs16k}
% 	\end{table}
%
% 	\begin{table}[h]
% 	\hspace*{-0.2cm}
% 	\small
% 	\centering
% 	\begin{tabular}{ll}
% 	\toprule
% 	\textbf{LHC16l - PCM Good Runs PCM INT7} \\ \midrule
% 		258962, 258964, 259088, 259090, 259091, 259096, 259099, 259117, 259118, 259162, \\ \midrule
% 		259164, 259204, 259257, 259261, 259263, 259264, 259269, 259270, 259271, 259272, \\ \midrule
% 		259273, 259274, 259302, 259303, 259305, 259307, 259334, 259336, 259339, 259340, \\ \midrule
% 		259341, 259342, 259378, 259382, 259388, 259389, 259394, 259395, 259396, 259473, \\ \midrule
% 		259477, 259649, 259650, 259668, 259697, 259700, 259703, 259704, 259705, 259711, \\ \midrule
% 		259713, 259747, 259748, 259750, 259751, 259752, 259756, 259781, 259788, 259789, \\ \midrule
% 		259822, 259841, 259842, 259860, 259866, 259867, 259868, 259888 \\
% 		\bottomrule
% 		\end{tabular}
% 		\caption{Runs used in this analysis for LHC16x, minimum bias trigger INT7 in PCM. The set of runs is identical for data and corresponding Monte Carlo productions that are used.}
% 		\label{tab:runs16l}
% 	\end{table}
%
% 	\begin{table}[h]
% 	\hspace*{-0.2cm}
% 	\small
% 	\centering
% 	\begin{tabular}{ll}
% 	\toprule
% 	\textbf{LHC16o - PCM Good Runs PCM INT7} \\ \midrule
% 		262395, 262396, 262397, 262398, 262399, 262418, 262419, 262422, 262423, 262424, \\ \midrule
% 		262425, 262426, 262428, 262430, 262487, 262489, 262490, 262492, 262563, 263331, \\ \midrule
% 		263813, 263814, 263823, 263824, 263829, 263830, 263923, 263979 \\
% 		\bottomrule
% 		\end{tabular}
% 		\caption{Runs used in this analysis for LHC16x, minimum bias trigger INT7 in PCM. The set of runs is identical for data and corresponding Monte Carlo productions that are used.}
% 		\label{tab:runs16o}
% 	\end{table}
%
% 	\begin{table}[h]
% 	\hspace*{-0.2cm}
% 	\small
% 	\centering
% 	\begin{tabular}{ll}
% 	\toprule
% 	\textbf{LHC16p - PCM Good Runs PCM INT7} \\ \midrule
% 		264076, 264078, 264082, 264085, 264086, 264109, 264110, 264129, 264137, 264138, \\ \midrule
% 		264139, 264164, 264188, 264190, 264194, 264197, 264198, 264232, 264233, 264235, \\ \midrule
% 		264238, 264259, 264260, 264261, 264262, 264264, 264265, 264266, 264267, 264273, \\ \midrule
% 		264277, 264279, 264281, 264305, 264306, 264312, 264336, 264341, 264345, 264346, \\ \midrule
% 		264347 \\
% 		\bottomrule
% 		\end{tabular}
% 		\caption{Runs used in this analysis for LHC16x, minimum bias trigger INT7 in PCM. The set of runs is identical for data and corresponding Monte Carlo productions that are used.}
% 		\label{tab:runs16p}
% 	\end{table}
%

\clearpage
\section{Quality Assurance on Cell Level}
\label{chap:extCellQA}
\subsection{Periodwise Cell QA}
The periodwise quality assurance on cell level is being performed using a multiple set of cuts on the following observables:\vspace{-0.2cm}
\begin{itemize}
\item Integrated occurrences of cells in events, see \hyperref[fig:cellHotCells]{Figure~\ref*{fig:cellHotCells}} \vspace{-0.3cm}
\item Cell energy distribution - mean / sigma, see \hyperref[fig:cellEnergy]{Figure~\ref*{fig:cellEnergy}} \vspace{-0.3cm}
\item Cell time distribution - mean / sigma, see \hyperref[fig:cellTime]{Figure~\ref*{fig:cellTime}} \vspace{-0.3cm}
\item Cell time distribution - ratio of fired cells within $20~ns$ to $500~ns$, see \hyperref[fig:cellHotCellsTime1D]{Figure~\ref*{fig:cellHotCellsTime1D}} \vspace{-0.3cm}
\item Cell energy distribution - ratio of fired cells above threshold to below threshold, see \hyperref[fig:cellHotCells2D]{Figure~\ref*{fig:cellHotCells2D}} \vspace{-0.2cm}
\end{itemize}
Only cells which fulfill all mentioned cuts are accepted and being used in the clusterization process. If a cell has been found to be bad in one of the observables, it has been added to the bad cell lists in OADB/OCDB so that it will be automatically removed from all events before entering any analysis. The following figures contain the corresponding plots for LHC16i, which serves as an example. All data and Monte Carlo periods from this analysis have been checked.

% \begin{figure}[h]
%  	\centering
%  	\includegraphics[width=0.6\textwidth]{figures/ClusterQA/Cells/CellHotCells\_LHC16i\_pass1\_block1.pdf}
%  	\vspace{-0.3cm}
%  	\caption{This plot contains the number of occurrences for each cell summed over all events before applying any cuts in the analysis. If a cell is only hot but does not fire with higher energies, so that Figure \hyperref[fig:cellEnergy]{Figure~\ref*{fig:cellEnergy}} cannot detect the bad cell, this plot has the capability to cut away those noisy cells. The applied cuts are represented by the dotted lines, only cells within are kept.}
%  	\label{fig:cellHotCells}
%  \end{figure}
%
%  \begin{figure}[h]
%  	\vspace{-0.1cm}
% 	\centering
% 	\includegraphics[width=0.6\textwidth]{figures/ClusterQA/Cells/CellEnergyVsSigma\_LHC16i\_pass1\_block1.pdf}
% 	\vspace{-0.3cm}
% 	\caption{For each cell, the mean cell energy is plotted vs sigma energy in this histogram. If cell energy distribution has multiple peaks or if a cell is hot and firing very often with high energy, it will be clearly separated from the good cells which accumulate very well. The cuts are chosen so that no Monte Carlo cells would be cut. The applied cuts are represented by the dotted lines, only cells within are kept.}
% 	\label{fig:cellEnergy}
% \end{figure}
%
% \begin{figure}
% 	\centering
% 	\includegraphics[width=0.6\textwidth]{figures/ClusterQA/Cells/CellTimeVsSigma\_LHC16i\_pass1\_block1.pdf}
% 	\vspace{-0.3cm}
% 	\caption{For each cell, the mean cell time is plotted vs sigma time in this histogram. If cell time distribution has multiple peaks or if a cell is firing very often at random times, it will be clearly separated from the good cells which accumulate very well. The applied cuts are represented by the dotted lines, only cells within are kept.}
% 	\label{fig:cellTime}
% \end{figure}
%
% \begin{figure}[h]
% 	\centering
% 	\includegraphics[width=0.6\textwidth]{figures/ClusterQA/Cells/CellHotCellsTime1D\_LHC16i\_pass1\_block1.pdf}
% 	\vspace{-0.3cm}
% 	\caption{The cell time distribution is taken as basis and is being integrated from $-20~ns$ to $20~ns$. Then the cell time distribution is integrated over the whole time axis. The ratio of both quantities is plotted in this figure. Ratios over 1.5 indicate that the cell primarily fires at basically random times which is the reason to exclude those cells. The applied cuts are represented by the dotted lines, only cells within are kept.}
% 	\label{fig:cellHotCellsTime1D}
% \end{figure}
%
% \begin{figure}[h]
% 	\centering
% 	\includegraphics[width=0.6\textwidth]{figures/ClusterQA/Cells/CellHotCells2D\_LHC16i\_pass1\_block1.pdf}
% 	\vspace{-0.3cm}
% 	\caption{The cell energy distribution is integrated from $0$ to $L$ and then from $L$ to $30~GeV$. The ratios are plotted for different values of $L$ in this figure. If a cell energy distribution has multiple peaks or a very different shape than other cells, it will be clearly visible in this plot at higher ratios. The dotted lines represent the cuts that have been used to exclude bad cells while only cells with lower ratios as the dotted lines are kept.}
% 	\label{fig:cellHotCells2D}
% \end{figure}
%
% \clearpage

\subsection{Runwise Cell QA}
%
% A detailed quality assurance has also been performed on runwise level. In this case, dead/cold and hot cells are identified on a run by run basis and then being marked as bad cells in OADB/OCDB, accordingly. In \hyperref[fig:cellHotCells]{Figure~\ref*{fig:cellHotCells}}, there is a typical energy fraction plot shown which is used to identify dead/cold and hot cells. The cell ID is plotted on x-axis while the y-axis contains the summed cell energy fraction of the cluster energy, summed over all cluster in a event and over all used events. As it turned out, the summed cell energy fraction is a much better discriminator than just looking at the integrated number of occurrences of respective cell IDs in clusters, which, in principle, could be used as well for the bad cell identification.
%
% %  		\begin{figure}[h]
% % 			\centering
% % 			\includegraphics[width=0.75\textwidth]{figures/ClusterQA/Cells/LHC12a\_EFrac\_Cells\_before.eps}
% % 			\caption{The cell ID is plotted versus the cell energy fraction of clusters, summed over all clusters in a event and over all events used in the analysis. One can clearly see some cold and hot cells by eye. The plot has been produced before the removal of bad cell candidates that have been identified.}
% % 			\label{fig:}
% %  	\end{figure}
%
% To be able to identify dead/cold and hot cells, some reference value, $\overline{E}\_{\mbox{\tiny frac, ref}}^{\mbox{\tiny X}}$, has to be computed in order to make the decision if cell at ID 'X' may be bad or not. This value is obtained by averaging the summed energy fraction over the previous 100 good cells.
% \begin{equation}
% \overline{E}\_{\mbox{\tiny frac, ref}}^{\mbox{\tiny X}} = \frac{1}{100}\cdot\sum\_{j=A-100}^{A-1} E\_{\mbox{\tiny frac}}^{\mbox{\tiny j-th cell ID}}
% \end{equation}
% This average is then compared with cell 'X' in order to decide if it is bad or not. For the first 100 cells, it is not possible, obviously, to compute the average using the given formula. For those cells at the beginning, a special procedure is being performed. The average of the first 100 cells is computed and used for the determination. To remove the possible occurrence bad cells in this average, the list is reiterated and the following cells are discarded: if cells are included with an energy fraction of zero or if a cell has double of the mean energy fraction and at least an energy fraction of '10'. Then the next cell IDs, beginning from 101, are included in the reference value for the first 100 cell IDs.
%
% The following criteria apply when identifying bad cells on runwise level:
% \begin{itemize}
% \item []\textbf{Dead cells}:\\
% A cell with ID 'X' is declared as dead in a run, if
% \begin{equation}
% E\_{\mbox{\tiny frac}}^{\mbox{\tiny X}} = 0
% \end{equation}
% and if it is not dead in anchored Monte Carlo productions, according to the given definition, in order to identify only new dead cells which are not yet included in OADB/OCDB.
% \item []\textbf{Cold cells}:\\
% A cell with ID 'X' is declared as cold in a run, if
% \begin{eqnarray}
% \mbox{if}(E\_{\mbox{\tiny frac}}^{\mbox{\tiny X}} \geq 80)& E\_{\mbox{\tiny frac}}^{\mbox{\tiny X}} < \overline{E}\_{\mbox{\tiny frac, ref}}^{\mbox{\tiny X}}/3 \\
% \mbox{else if}(E\_{\mbox{\tiny frac}}^{\mbox{\tiny X}} \geq 40)& E\_{\mbox{\tiny frac}}^{\mbox{\tiny X}} < \overline{E}\_{\mbox{\tiny frac, ref}}^{\mbox{\tiny X}}/5 \\
% \mbox{else if}(E\_{\mbox{\tiny frac}}^{\mbox{\tiny X}} \geq 10)& E\_{\mbox{\tiny frac}}^{\mbox{\tiny X}} < \overline{E}\_{\mbox{\tiny frac, ref}}^{\mbox{\tiny X}}/8 \\
% \mbox{else if}(E\_{\mbox{\tiny frac}}^{\mbox{\tiny X}} < 10)& E\_{\mbox{\tiny frac}}^{\mbox{\tiny X}} < \overline{E}\_{\mbox{\tiny frac, ref}}^{\mbox{\tiny X}}/10
% \end{eqnarray}
% and if it is not cold in anchored Monte Carlo productions, according to the given definition, in order to identify only true cold cells.
%
% \item []\textbf{Hot cells}:\\
% A cell with ID 'X' is declared as hot in a run, if
% \begin{eqnarray}
% \mbox{if}(E\_{\mbox{\tiny frac}}^{\mbox{\tiny X}} > 80)& E\_{\mbox{\tiny frac}}^{\mbox{\tiny X}} < 2 \cdot \overline{E}\_{\mbox{\tiny frac, ref}}^{\mbox{\tiny X}} \\
% \mbox{else if}(E\_{\mbox{\tiny frac}}^{\mbox{\tiny X}} > 20)& E\_{\mbox{\tiny frac}}^{\mbox{\tiny X}} < 3 \cdot \overline{E}\_{\mbox{\tiny frac, ref}}^{\mbox{\tiny X}} \\
% \mbox{else if}(E\_{\mbox{\tiny frac}}^{\mbox{\tiny X}} > 8)& E\_{\mbox{\tiny frac}}^{\mbox{\tiny X}} < 4 \cdot \overline{E}\_{\mbox{\tiny frac, ref}}^{\mbox{\tiny X}} \\
% \mbox{else if}(E\_{\mbox{\tiny frac}}^{\mbox{\tiny X}} > 5)& E\_{\mbox{\tiny frac}}^{\mbox{\tiny X}} < 5 \cdot \overline{E}\_{\mbox{\tiny frac, ref}}^{\mbox{\tiny X}}
% \end{eqnarray}
% and if it is not hot in anchored Monte Carlo productions, according to the given definition, in order to identify only true hot cells.
% \end{itemize}
%
% While looping through all cell IDs, the cells, that have been identified to belong to one of the three just described categories, will not enter the ongoing average determination for the upcoming cell IDs.
%
% For the dead and cold cell determination, only runs are inspected that have at least $1\cdot10^5$ fired cells integrated over the whole run. For the hot cell determination, the threshold to inspect a run is $1\cdot10^4$ to ensure enough statistics to perform a meaningful selection. If there are clear sub-runranges where some cells are bad, they will be marked as bad in those runs including the skipped runs due to statistics. If a cell is suspicious in the majority of runs, it will be directly excluded from the whole period.
% \clearpage

\section{Additional Cluster/Photon QA Plots}
\label{chap:QAClusterShape}
\label{chap:extQALHC16}

\hyperref[fig:clusterM02]{Figure~\ref*{fig:clusterM02}} shows the cluster shape parameter $\lambda\_{0}^{2}$ including a comparison of data and Monte Carlo.
Moreover, runwise distributions are also shown for all productions.
The corresponding cluster shape parameter $\lambda\_{1}^{2}$ can be found in \hyperref[fig:clusterM20]{Figure~\ref*{fig:clusterM20}}, including a comparison of data and Monte Carlo.
Additionally, runwise distributions are shown for data and both minimum bias simulations for the example period LHC12b.

% \begin{figure}[h]
% 		\centering
% 		\includegraphics[width=0.44\textwidth]{figures/TriggerQA/M02\_afterQA.eps}
% 		\hspace{0.6cm}
% 		\includegraphics[width=0.48\textwidth]{figures/ClusterQA/Runwise/LHC12b/LHC12b/ClusterM02\_Runwise\_LHC12b.eps}\\
% 		\includegraphics[width=0.48\textwidth]{figures/ClusterQA/Runwise/LHC12b/LHC15h1b/ClusterM02\_Runwise\_Pythia8.eps}
% 		\includegraphics[width=0.48\textwidth]{figures/ClusterQA/Runwise/LHC12b/LHC15h2b/ClusterM02\_Runwise\_Phojet.eps}
% 		\caption{(top left) Cluster shape parameter $\lambda\_{0}^{2}$ for \acs{EMCal} clusters for data, Pythia8, Phojet and JetJet Monte Carlo. All points are scaled by the number of available events in each production. Only parameters between 0.1 and 0.5 are accepted as it can be taken from \hyperref[tab:clusterSelection]{Table~\ref*{tab:clusterSelection}}. The other three histograms show runwise superpositions of the $\lambda\_{0}^{2}$ distributions for data, Pythia8 and Phojet that are also scaled by the number of available events in each run. }
% 		\label{fig:clusterM02}
%  \end{figure}
%  \begin{figure}[h]
% 		\centering
% 		\includegraphics[width=0.44\textwidth]{figures/TriggerQA/M20\_afterQA.eps}
% 		\hspace{0.6cm}
% 		\includegraphics[width=0.48\textwidth]{figures/ClusterQA/Runwise/LHC12b/LHC12b/ClusterM20\_Runwise\_LHC12b.eps}\\
% 		\includegraphics[width=0.48\textwidth]{figures/ClusterQA/Runwise/LHC12b/LHC15h1b/ClusterM20\_Runwise\_Pythia8.eps}
% 		\includegraphics[width=0.48\textwidth]{figures/ClusterQA/Runwise/LHC12b//LHC15h2b/ClusterM20\_Runwise\_Phojet.eps}
% 		\caption{(top left) Cluster shape parameter $\lambda\_{1}^{2}$ for \acs{EMCal} clusters in data, Pythia8, Phojet and JetJet Monte Carlo. All points are scaled by the number of available events in each production. As it can be taken from \hyperref[tab:clusterSelection]{Table~\ref*{tab:clusterSelection}}, there is only a cut on $\lambda\_{0}^{2}$. Since $\lambda\_{1}^{2}$ and $\lambda\_{0}^{2}$ are correlated the possible values of $\lambda\_{1}^{2}$ are being limited to 0.5. The other three histograms show runwise superpositions of the $\lambda\_{1}^{2}$ distributions for data, Pythia 8 and Phojet that are also scaled by the number of available events in each run. }
% 		\label{fig:clusterM20}
% \end{figure}

The data points follow Monte Carlo equivalents reasonably well, which nicely agree to each other and all runs show the same distributions.
Both the average and the standard deviation of the shower shape parameters $\lambda\_{0}^{2}$ and $\lambda\_{1}^{2}$ stay constant over all available runs as it can be seen in \hyperref[fig:QA-M02]{Figure~\ref*{fig:QA-M02}} and \hyperref[fig:QA-M20]{Figure~\ref*{fig:QA-M20}}.
Both Monte Carlo productions agree while again data is at slightly higher values.
This behavior makes us confident that the cluster shape is reproduced reasonably well in Monte Carlo.

Finally, in \hyperref[fig:clusterM02M20]{Figure~\ref*{fig:clusterM02M20}} a 2-dimensional plot of both shape parameters $\lambda\_{1}^{2}$ and $\lambda\_{0}^{2}$ can be found.

% 	\begin{figure}[h]
%			\centering
%			\includegraphics[width=0.48\textwidth]{figures/ClusterQA/Runwise/hClusterM02-Mean.eps}
%			\includegraphics[width=0.48\textwidth]{figures/ClusterQA/Runwise/hClusterM02-RMS.eps}
%			\caption{(left) Average value of shower shape parameter $\lambda\_{0}^{2}$. (right) Standard deviation of $\lambda\_{0}^{2}$.}
%			\label{fig:QA-M02}
%  	\end{figure}
%
% 	 \begin{figure}[h]
%			\centering
%			\includegraphics[width=0.48\textwidth]{figures/ClusterQA/Runwise/hClusterM20-Mean.eps}
%			\includegraphics[width=0.48\textwidth]{figures/ClusterQA/Runwise/hClusterM20-RMS.eps}
%			\caption{(left) Average value of shower shape parameter $\lambda\_{1}^{2}$. (right) Standard deviation of $\lambda\_{1}^{2}$.}
%			\label{fig:QA-M20}
%  	\end{figure}
%
% \begin{figure}[h]
%		\centering
%		\includegraphics[width=0.5\textwidth]{figures/ClusterQA/M20VsM02\_all\_LHC12.eps}
%		\caption{A two dimensional plot of $\lambda\_{1}^{2}$ vs. $\lambda\_{0}^{2}$ is shown here. The most common cluster shapes can be well distinguished as they form different peaks with very different shapes in the histogram. Since we only use very loose cuts and do not explicitly use cluster shape analysis to enhance the photon sample, no further discussions on that topic are made in this note.}
%		\label{fig:clusterM02M20}
%\end{figure}
%
%\begin{figure}[h]
%		\centering
%		\includegraphics[width=0.48\textwidth]{figures/TriggerQA/ratios/ratio\_Energy\_Cluster\_afterQA.eps}
%		\includegraphics[width=0.48\textwidth]{figures/TriggerQA/ratios/ratio\_NCells\_afterQA.eps}
%		\caption{Concerning \hyperref[fig:clusterEnergy]{Figures~\ref*{fig:clusterEnergy}} and \hyperref[fig:clusterNCells]{\ref*{fig:clusterNCells}}, the corresponding ratios can be found in the left and right plot accordingly.}
%		\label{fig:clusterQAratios}
%\end{figure}
%
%\begin{figure}[h]
%		\centering
%		\includegraphics[width=0.48\textwidth]{figures/TriggerQA/ratios/ratio\_NGoodTracks.eps}
%		\includegraphics[width=0.48\textwidth]{figures/TriggerQA/ratios/ratio\_GammaCandidates\_EMCAL.eps}
%		\caption{Ratios of charged track multiplicities (left) and \acs{EMCal} ratios of cluster multiplicities (right) for LHC12a-i with the corresponding minimum bias productions and JetJet MC productions as well.}
%		\label{fig:clusterQAratios2}
%\end{figure}
%
%\begin{figure}[h]
%		\centering
%		\includegraphics[width=0.48\textwidth]{figures/TriggerQA/ratios/ratio\_NGoodTracks\_EMC7.eps}
%		\includegraphics[width=0.48\textwidth]{figures/TriggerQA/ratios/ratio\_NGoodTracks\_EGA.eps}
%		\caption{Ratios of charged track multiplicities for EMC L0, INT7 (left) and for EMC L1-GA, INT7 (right).}
%		\label{fig:clusterQAratios3}
%\end{figure}
%
%\begin{figure}[h]
%		\centering
%		\includegraphics[width=0.48\textwidth]{figures/PhotonQA/Comparison/Ratios/ratio\_Electron\_FindTPCClusters.eps}
%		\includegraphics[width=0.48\textwidth]{figures/PhotonQA/Comparison/Ratios/ratio\_Positron\_FindTPCClusters.eps}\\							\includegraphics[width=0.48\textwidth]{figures/PhotonQA/Comparison/Ratios/ratio\_Electron\_TPCClusters.eps}
%		\includegraphics[width=0.48\textwidth]{figures/PhotonQA/Comparison/Ratios/ratio\_Positron\_TPCClusters.eps}
%		\caption{Ratios of \hyperref[fig:TPCfindableClusters]{Figure~\ref*{fig:TPCfindableClusters}} and \hyperref[fig:TPCClusters]{Figure~\ref*{fig:TPCClusters}}.}
%		\label{fig:photonQAratios}
%\end{figure}
%
%\begin{figure}[h]
%		\centering
%		\includegraphics[width=0.48\textwidth]{figures/TriggerQA/EMC7/ratio\_GammaCandidates\_EMCAL.eps}
%		\includegraphics[width=0.48\textwidth]{figures/TriggerQA/EGA/ratio\_GammaCandidates\_EMCAL.eps}
%		\caption{Ratios of \acs{EMCal} cluster multiplicities for EMC L0, INT7 (left) and for EMC L1-GA, INT7 (right).}
%		\label{fig:clusterQAratios5}
%\end{figure}
%
%\begin{figure}[h]
%		\centering
%			\includegraphics[width=0.49\textwidth]{figures/ClusterQA/EtaVsPhi\_RatioLHC15h1\_LHC12\_afterClusterQA.eps}
%			\includegraphics[width=0.49\textwidth]{figures/ClusterQA/EtaVsPhi\_RatioLHC15h2\_LHC12\_afterClusterQA.eps}\\
%			\includegraphics[width=0.49\textwidth]{figures/TriggerQA/EtaVsPhi\_RatioLHC16c2\_LHC12ci\_afterClusterQA.eps}
%		\caption{Ratios of \hyperref[fig:clusterEtaPhi]{Figure~\ref*{fig:clusterEtaPhi}}.}
%		\label{fig:clusterQAratios4}
%\end{figure}
%
%\begin{figure}[h]
%		\centering
%			\includegraphics[width=0.49\textwidth]{figures/EventQA/Comparison/Vertex_Z.eps}
%			\includegraphics[width=0.49\textwidth]{figures/EventQA/Comparison/Ratios/ratio_Vertex_Z.eps}
%		\caption{Z-vertex positions for data and minimum bias Monte Carlo compared for all accepted events. Furthermore, ratios to data are shown.}
%		\label{fig:eventQAzvertex}
%\end{figure}
%
%\clearpage
%
%\begin{figure}[h]
%	\centering
%	\includegraphics[width=0.94\textwidth]{figures/ClusterQA/Runwise/LHC12b/LHC12b/hClusterEtaPhi\_scaledNEventsAndMean\_p0.eps}\\
%	\includegraphics[width=0.94\textwidth]{figures/ClusterQA/Runwise/LHC12b/LHC12b/hClusterEtaPhi\_scaledNEventsAndMean\_p1.eps}
%	\caption{Runwise \acs{EMCal} cluster $\eta$-$\phi$ distributions for data, given the example of LHC12b. The number of available events for each single run can be found in left part of figure \ref{fig:QA-NClusters}.}
%	\label{runwiseEtaPhi:data}
%\end{figure}
%\begin{figure}[h]
%	\centering
%	\includegraphics[width=0.94\textwidth]{figures/ClusterQA/Runwise/LHC12b/LHC15h1b/hClusterEtaPhi\_scaledNEventsAndMean\_p0.eps}\\
%	\includegraphics[width=0.94\textwidth]{figures/ClusterQA/Runwise/LHC12b/LHC15h1b/hClusterEtaPhi\_scaledNEventsAndMean\_p1.eps}
%	\caption{Runwise \acs{EMCal} cluster $\eta$-$\phi$ distributions for Pythia8, according to example run list of LHC12b. The number of available events for each single run can be found in left part of figure \ref{fig:QA-NClusters}.}
%	\label{runwiseEtaPhi:pythia}
%\end{figure}
%
%\newpage
%
%\begin{figure}[h]
%	\centering
%	\includegraphics[width=0.94\textwidth]{figures/ClusterQA/Runwise/LHC12b/LHC15h2b/hClusterEtaPhi\_scaledNEventsAndMean\_p0.eps}\\
%	\includegraphics[width=0.94\textwidth]{figures/ClusterQA/Runwise/LHC12b/LHC15h2b/hClusterEtaPhi\_scaledNEventsAndMean\_p1.eps}
%	\caption{Runwise \acs{EMCal} cluster $\eta$-$\phi$ distributions for Phojet, according to example run list of LHC12b. The number of available events for each single run can be found in left part of figure \ref{fig:QA-NClusters}.}
%	\label{runwiseEtaPhi:phojet}
%\end{figure}
%\clearpage
%
% 	 \begin{figure}[h]
%			\centering
%			\includegraphics[width=0.48\textwidth]{figures/ClusterQA/ClusterEnergyVsModule\_LHC12.eps}
%			\includegraphics[width=0.48\textwidth]{figures/ClusterQA/ClusterEnergyVsModule\_Projected\_LHC12.eps}
%			\caption{On the left side, cluster energies are plotted versus supermodule number for INT7. The z-axis displays the number of occurrences of the respective cluster energies for each supermodule summed over all events. On the right side, the projection to the x-axis is shown in order to being able to directly compare the supermodules.}
% 	\end{figure}
%
% 	 \begin{figure}[h]
%			\centering
%			\includegraphics[width=0.48\textwidth]{figures/TriggerQA/EMC7/ClusterEnergyVsModule\_LHC12.eps}
%			\includegraphics[width=0.48\textwidth]{figures/TriggerQA/EMC7/ClusterEnergyVsModule\_Projected\_LHC12.eps}
%			\caption{On the left side, cluster energies are plotted versus supermodule number for EMC L0, INT7. The z-axis displays the number of occurrences of the respective cluster energies for each supermodule summed over all events. On the right side, the projection to the x-axis is shown in order to being able to directly compare the supermodules.}
% 	\end{figure}
%
% 	 	 \begin{figure}[h]
%			\centering
%			\includegraphics[width=0.48\textwidth]{figures/TriggerQA/EGA/ClusterEnergyVsModule\_LHC12.eps}
%			\includegraphics[width=0.48\textwidth]{figures/TriggerQA/EGA/ClusterEnergyVsModule\_Projected\_LHC12.eps}
%			\caption{On the left side, cluster energies are plotted versus supermodule number for EMC L1-GA, INT7. The z-axis displays the number of occurrences of the respective cluster energies for each supermodule summed over all events. On the right side, the projection to the x-axis is shown in order to being able to directly compare the supermodules.}
% 	\end{figure}
%
%     \begin{figure}[h]
%			\centering
%			\includegraphics[width=0.48\textwidth]{figures/ClusterQA/ModuleEnergyVsModule\_LHC12.eps}
%			\includegraphics[width=0.48\textwidth]{figures/ClusterQA/ModuleEnergyVsModule\_Projected\_LHC12\_LOG.eps}
%			\caption{On the left side, the total energy, which is being registered within a supermodule by summing over all energies of all available cells in an event, are plotted versus supermodule number. The z-axis displays the number of occurrences of the respective total energies for each supermodule summed over all events. On the right side, the projection to the x-axis is shown in order to being able to directly compare the supermodules.}
% 	\end{figure}
%
%    \begin{figure}[h]
%			\centering
%			\includegraphics[width=0.48\textwidth]{figures/ClusterQA/NCellsAbove100VsModule\_LHC12.eps}
%			\includegraphics[width=0.48\textwidth]{figures/ClusterQA/NCellsAbove100VsModule\_Projected\_LHC12.eps}
%			\caption{On the left side, the number of cells above 100 MeV, which are being registered within a supermodule in an event, are plotted versus supermodule number. The z-axis displays the number of occurrences of the respective total number of cells for each supermodule summed over all events. On the right side, the projection to the x-axis is shown in order to being able to directly compare the supermodules.}
% 	\end{figure}
%
%   \begin{figure}[h]
%			\centering
%			\includegraphics[width=0.48\textwidth]{figures/ClusterQA/NCellsAbove1500VsModule\_LHC12.eps}
%			\includegraphics[width=0.48\textwidth]{figures/ClusterQA/NCellsAbove1500VsModule\_Projected\_LHC12.eps}
%			\caption{On the left side, the number of cells above 1500 MeV, which are being registered within a supermodule in an event, are plotted versus supermodule number. The z-axis displays the number of occurrences of the respective total number of cells for each supermodule summed over all events. On the right side, the projection to the x-axis is shown in order to being able to directly compare the supermodules.}
% 	\end{figure}
%
% 	 	 \begin{figure}[h]
%			\centering
%			\includegraphics[width=0.47\textwidth]{figures/TriggerQA/Compare/ClusterEnergy.eps}
%			\includegraphics[width=0.51\textwidth]{figures/TriggerQA/Compare/hClusterEnergy-Mean.eps}
%			\caption{(left) Energy distributions of \acs{EMCal} clusters for minimum bias and triggered data are shown here. (right) This plot displays runwise the mean energy of \acs{EMCal} clusters for all available runs.}
%			\label{fig:clusterEnergyTrigger}
% 	\end{figure}
%
% 	\begin{figure}[h]
%			\centering
%			\includegraphics[width=0.47\textwidth]{figures/TriggerQA/Compare/NCells.eps}
%			\includegraphics[width=0.51\textwidth]{figures/TriggerQA/Compare/hClusterNCells-Mean.eps}
%			\caption{The left histogram shows the number of cells of \acs{EMCal} clusters for minimum bias and triggered data. Following on the right side, the respective runwise mean number of cells per cluster is displayed for the different triggers.}
%			\label{fig:clusterNCellsTrigger}
% 	\end{figure}
%
% 	\begin{figure}[h]
%			\centering
%			\includegraphics[width=0.47\textwidth]{figures/TriggerQA/Compare/M02.eps}
%			\includegraphics[width=0.51\textwidth]{figures/TriggerQA/Compare/hClusterM02-Mean.eps}
%			\caption{(left) Cluster shape parameter $\lambda\_{0}^{2}$ for \acs{EMCal} clusters for minimum bias and triggered data. All points are scaled by the number of available events in each production. Only parameters between 0.1 and 0.5 are accepted as it can be taken from \hyperref[tab:clusterSelection]{Table~\ref*{tab:clusterSelection}}. The other histogram shows the runwise superposition of the $\lambda\_{0}^{2}$ distributions for triggered data.}
%			\label{fig:clusterM02Trigger}
% 	\end{figure}
%
% 	 \begin{figure}[h]
%			\centering
%			\includegraphics[width=0.47\textwidth]{figures/TriggerQA/Compare/M20.eps}
%			\includegraphics[width=0.51\textwidth]{figures/TriggerQA/Compare/hClusterM20-Mean.eps}
%			\caption{(left) Cluster shape parameter $\lambda\_{1}^{2}$ for \acs{EMCal} clusters for minimum bias and triggered data. All points are scaled by the number of available events in each production. As it can be taken from \hyperref[tab:clusterSelection]{Table~\ref*{tab:clusterSelection}}, there is only a cut on $\lambda\_{0}^{2}$. Since $\lambda\_{1}^{2}$ and $\lambda\_{0}^{2}$ are correlated the possible values of $\lambda\_{1}^{2}$ are being limited to 0.5. The other histogram shows the runwise superposition of the $\lambda\_{1}^{2}$ distributions for triggered data.}
%			\label{fig:clusterM20Trigger}
% 	\end{figure}
%
% 	\begin{figure}[h]
%			\centering
%			\includegraphics[width=0.49\textwidth]{figures/TriggerQA/Compare/hCluster-FractionMatches.eps}
%			\includegraphics[width=0.49\textwidth]{figures/TriggerQA/Compare/hCluster-FractionMatchesH.eps}
%			\caption{(left) The fraction of matches is shown for each single run used in this analysis for minimum bias and triggered data respectively. The fraction is calculated by dividing the number of successful cluster -- V$^{0}$-matches by the total number of available mother candidates.(right) The fraction of matches is shown for $\it{p}\_{\text{T, Pair}}>2.5$~GeV/$\it{c}$ for each single run used in this analysis for minimum bias and triggered data respectively. The fraction is calculated by dividing the number of successful cluster -- V$^{0}$-matches above the $\it{p}\_{\text{T}}$-threshold by the total number of available mother candidates in that interval.}
%			\label{fig:QA-FractionMatchesTrigger}
% 	\end{figure}
%
% 	\begin{figure}[h]
%			\centering
%			\includegraphics[width=0.45\textwidth]{figures/TriggerQA/Compare/dEta.eps}
%			\includegraphics[width=0.45\textwidth]{figures/TriggerQA/Compare/dPhi.eps}\\
%			\includegraphics[width=0.45\textwidth]{figures/TriggerQA/Compare/dEtaMatched.eps}
%			\includegraphics[width=0.45\textwidth]{figures/TriggerQA/Compare/dPhiMatched.eps}
%			\caption{(top) $\Delta\eta$- and $\Delta\phi$-projections of \hyperref[fig:matchingAllAccepted]{Figure~\ref*{fig:matchingAllAccepted}} for minimum bias and triggered data. (bottom) For minimum bias and triggered data, only the projections of successful matches are shown. All distributions in this figure are respectively scaled to an integral of 1.}
%			\label{fig:matchingDeltaTrigger}
% 	\end{figure}
%
% 	\begin{figure}[h]
%			\centering
%			\includegraphics[width=0.48\textwidth]{figures/TriggerQA/Compare/hClusterDeltaEta-Mean.eps}
%			\includegraphics[width=0.48\textwidth]{figures/TriggerQA/Compare/hClusterDeltaEta-RMS.eps}\\
%			\includegraphics[width=0.48\textwidth]{figures/TriggerQA/Compare/hClusterDeltaPhi-Mean.eps}
%			\includegraphics[width=0.48\textwidth]{figures/TriggerQA/Compare/hClusterDeltaPhi-RMS.eps}\\
%			\caption{On the left side, the average values of $\Delta\eta$ and $\Delta\phi$ are plotted for every run for triggered data while the two histograms on the right side contain the mean standard deviations of the respective observables. All points are obtained by looking at the $\Delta\eta$- and $\Delta\phi$-projections of successful matches.}
%			\label{fig:QA-DeltaEtaPhiTrigger}
% 	\end{figure}
%
% 	\begin{figure}[h]
%			\centering
%			\includegraphics[width=0.48\textwidth]{figures/TriggerQA/Compare/MotherPi0Conv\_Eta.eps}
%			\includegraphics[width=0.48\textwidth]{figures/TriggerQA/Compare/MotherPi0Conv\_Phi.eps}
%			\caption{(left) Comparison of minimum bias and triggered data for $\eta$ distribution of selected conversion photon candidates under the $\pi^{0}$-peak. (right) Comparison of minimum bias and triggered data for $\phi$ distribution of selected conversion photon candidates under the $\pi^{0}$-peak. All histograms in both plots are scaled by the respective number of accepted events.}
%			\label{fig:etaphiPi0UnderpeakTrigger}
% 	\end{figure}
%
% 	\begin{figure}[h]
%			\centering
%			\includegraphics[width=0.48\textwidth]{figures/TriggerQA/Compare/MotherEtaConv\_Eta.eps}
%			\includegraphics[width=0.48\textwidth]{figures/TriggerQA/Compare/MotherEtaConv\_Phi.eps}
%			\caption{(left) Comparison of minimum bias and triggered data for $\eta$ distribution of selected conversion photon candidates under the $\eta$-peak. (right) Comparison of minimum bias and triggered data for $\phi$ distribution of selected conversion photon candidates under the $\eta$-peak. All histograms in both plots are scaled by the respective number of accepted events.}
%			\label{fig:etaphiEtaUnderpeakTrigger}
% 	\end{figure}

\clearpage


\section{Additional Neutral Meson Plots}
\label{chap:AppNeutralMeson}

% 	\begin{figure}[h]
% 		\centering
% 			\includegraphics[width=0.5\textwidth]{figures/Meson/MesonTrigger/Pi0_data_CorrectedYieldFinal_WithMC.eps}\\
% 			\includegraphics[width=0.6\textwidth]{figures/Meson/MesonTrigger/Pi0_data_RatioSpectraToFitFinal_withMC.eps}
% 			\caption{Final measured neutral pion transverse momentum spectra with \ac{EMCal} for \pp collisions at \sth compared to the Monte Carlo input spectra (blue). Indeed, the input spectrum is a realistic one with a reasonable power law behavior, so that it is suitable for this analysis.}
% 			\label{fig:EMCalWithMC}
% 		\end{figure}

% \clearpage

\subsection{\texorpdfstring{Neutral Mesons in \pp Collisions at \s~=~13~TeV, MB INT7}{Neutral Mesons in pp Collisions at 13 TeV, MB INT7}}

% 		\begin{figure}[h]
% 		 	\includegraphics[width=0.96\textwidth]	{figures/Meson/INT7new/Pi0_data_MesonWithBck_00010113_00200009327000008250400000_1111111067032230000_0163103100000010.eps}\\
% 			\includegraphics[width=0.96\textwidth]{figures/Meson/INT7new/Pi0_data_MesonSubtracted_00010113_00200009327000008250400000_1111111067032230000_0163103100000010.eps}
% 			\caption{Invariant-mass distribution of reconstructed photon pairs $M_{\gamma\gamma}$ around the neutral pion mass in \pT slices in pp collisions at \sth  before (upper plot) and after subtraction (lower plot) for minimum bias INT7 triggered data. The black histograms in the upper plot show the combined signal and background distribution and the blue histograms show the calculated and normalized mixed event background. After the subtraction, the invariant mass distributions are fitted with \hyperref[eq:gausexp]{Equation~\ref*{eq:gausexp}}, which is shown in green in the lower plot.}
% 			\label{fig:InvMassPi0PtSliceINT7}
% 		\end{figure}
%
% 		\begin{figure}
% 		 	\includegraphics[width=0.96\textwidth]	{figures/Meson/INT7new/Eta_data_MesonWithBck_00010113_00200009327000008250400000_1111111067032230000_0163103100000010.eps}\\
% 			\includegraphics[width=0.96\textwidth]{figures/Meson/INT7new/Eta_data_MesonSubtracted_00010113_00200009327000008250400000_1111111067032230000_0163103100000010.eps}
% 			\caption{Invariant-mass distribution of reconstructed photon pairs $M_{\gamma\gamma}$ around the eta meson mass in \pT slices in pp collisions at \sth  before (upper plot) and after subtraction (lower plot) for minimum bias INT7 triggered data. The black histograms in the upper plot show the combined signal and background distribution and the blue histograms show the calculated and normalized mixed event background. After the subtraction, the invariant mass distributions are fitted with \hyperref[eq:gausexp]{Equation~\ref*{eq:gausexp}}, which is shown in green in the lower plot.}
% 			\label{fig:InvMassEtaPtSliceINT7}
% 		\end{figure}
%
% 		\begin{figure}
% 		 	\includegraphics[width=0.96\textwidth]	{figures/Meson/INT7new/Pi0_MesonLineShapeCompared_00010113_00200009327000008250400000_1111111067032230000_0163103100000010.eps}\\
% 			\includegraphics[width=0.96\textwidth]{figures/Meson/INT7new/Eta_MesonLineShapeCompared_00010113_00200009327000008250400000_1111111067032230000_0163103100000010.eps}
% 			\caption{Invariant-mass distribution of reconstructed photon pairs $M_{\gamma\gamma}$ around the neutral pion (upper) and eta meson (lower) mass in \pT slices in pp collisions at \sth after subtraction. The black dots represent the distribution obtained after the background subtraction in the data, while the blue dots show the same distribution for the reconstructed Monte Carlo sample. On top of these, the real neutral pion and eta distributions, according to the Monte Carlo, are plotted as red histograms, those were obtained by validating that both reconstructed photons are actually photons (or decay products of one) and originate in the same meson, which has to be a neutral pion or eta meson. All distributions are normalized the integral in the given invariant mass range.}
% 			\label{fig:InvMassShapeComp2760}
% 		\end{figure}
%
% 		\begin{figure}
% 		 	\includegraphics[width=0.96\textwidth]{figures/Meson/INT7new/Pi0_MC_TrueMesonDecomposedPhotonsAndElectron_00010113_00200009327000008250400000_1111111067032230000_0163103100000010.eps}
% 		 	\includegraphics[width=0.96\textwidth]	{figures/Meson/INT7new/Pi0_MC_TrueMesonDecomposedMerged_00010113_00200009327000008250400000_1111111067032230000_0163103100000010.eps}
% 			\caption{Invariant-mass distribution of reconstructed validated photon pairs $M_{\gamma\gamma}$ around the mass of the neutral pion mass (0.135~GeV/$c^2$) in \pT slices in pp collisions at \sth. The upper distributions have been obtained by using the Monte Carlo simulation to validate that the two photons come from the same neutral meson (black bullets). Afterwards it has been checked whether one (purple) or both (cyan) of the reconstructed photons in the calorimeter converted or if both clusters were from a real photon (red).
% 			For the lower plots, the same convention applies for the black points. However, this time is was checked whether the cluster reconstructed in the \ac{EMCal} contains more than one particle from the same neutral pion. In this case, the red bullets show the distribution where there are at least two particles from the same neutral pion in the same cluster, while the blue distribution in addition requires that at least one of them was originating in a conversion.}
% 			\label{fig:InvMassPi0PtSliceMCDec2760}
% 		\end{figure}
% 		\begin{figure}
% 			\includegraphics[width=0.96\textwidth]{figures/Meson/INT7new/Eta_MC_TrueMesonDecomposedPhotonsAndElectron_00010113_00200009327000008250400000_1111111067032230000_0163103100000010.eps}
% 			\includegraphics[width=0.96\textwidth]	{figures/Meson/INT7new/Eta_MC_TrueMesonDecomposedMerged_00010113_00200009327000008250400000_1111111067032230000_0163103100000010.eps}
% 			\caption{Invariant-mass distribution of reconstructed validated photon pairs $M_{\gamma\gamma}$ around the mass of the $\eta$ meson mass (0.548~GeV/$c^2$) in \pT slices in pp collisions at \sth. The upper distributions have been obtained by using the Monte Carlo simulation to validate that the two photons come from the same neutral meson (black bullets). Afterwards it has been checked whether one (purple) or both (cyan) of the reconstructed photons in the calorimeter converted or if both clusters were from a real photon (red). For the lower plots, the same convention applies for the black points. However, this time is was checked whether the cluster reconstructed in the \ac{EMCal} contains more than one particle from the same eta meson. In this case, the red bullets show the distribution where there are at least two particles from the same eta meson in the same cluster, while the blue distribution in addition requires that at least one of them was originating in a conversion.}
% 			\label{fig:InvMassEtaPtSliceMCDec2760}
% 		\end{figure}
%
% 			\begin{figure}
% 			\includegraphics[width=0.49\textwidth]{figures/Meson/INT7new/Weighting/Pi0_data_CorrectedYield.eps}
% 			\includegraphics[width=0.49\textwidth]{figures/Meson/INT7new/Weighting/Eta_data_CorrectedYield.eps}
% 				\caption{Corrected neutral meson spectra for the minimum bias INT7 trigger in \sth. For both mesons (pion on left side, eta on right side), the fully corrected spectra are compared using the multiplicity weighted correction factors and the ones without any modification.}
% 			\label{fig:CompMultWeight}
% 			\end{figure}
%
% 			\begin{figure}
% 			\centering
% 			\includegraphics[width=1\textwidth]{figures/Meson/INT7new/Timing/Pi0_data_MesonSubtracted_00000113_1111111003032220000_0163103100000050.eps}
% 			\caption{This figure shows the background subtracted invariant mass distributions for an \acs{EMCal}-\acs{EMCal} analysis where one cluster was taken from within the cluster time cut window and paired to another cluster that had to be outside the timing cut window. One can clearly see that there is no yield at all visible and therefore, all fits are failing. Thus, the cluster time cut is fully efficient and there is no yield found due to contributions from outside the timing cut.}
% 			\label{fig:EMCALInOutTiming}
%  			\end{figure}
% 		\clearpage


		\subsection{Detailed Systematics}
		\label{sec:SysDetailed}

			\begin{figure}[t]
				\centering
				\includegraphics[width=0.49\textwidth]{figures/Meson/Systematics/EMC/SysMeanNewWithMean_Pi0_13TeVINT7.png}
				\includegraphics[width=0.49\textwidth]{figures/Meson/Systematics/EMC/SysMeanNewWithMean_Eta_13TeVINT7.png}
				\includegraphics[width=0.49\textwidth]{figures/Meson/Systematics/EMC/SysMeanNewWithMean_EtaToPi0_13TeVINT7.png}
				\caption{from the \ac{EMCal} method for \pp collisions at \s~=~13~TeV for the INT7 trigger. The colored points represent the individual error sources, while the black points represent the final systematic error for the $\pi^0$ meson transverse momentum spectrum.}
				\label{fig:SysErrEMCDetailed_INT7}
			\end{figure}
			\begin{figure}[t]
				\centering
			  	\includegraphics[width=0.49\textwidth]{figures/Meson/Systematics/EMC/SysMeanNewWithMean_Pi0_13TeVEG2.png}
			  	\includegraphics[width=0.49\textwidth]{figures/Meson/Systematics/EMC/SysMeanNewWithMean_Eta_13TeVEG2.png}
			  	\includegraphics[width=0.49\textwidth]{figures/Meson/Systematics/EMC/SysMeanNewWithMean_EtaToPi0_13TeVEG2.png}
			  	\caption{from the \ac{EMCal} method for \pp collisions at \s~=~13~TeV for the EG2 trigger. The colored points represent the individual error sources, while the black points represent the final systematic error for the $\pi^0$ meson transverse momentum spectrum.}
			  	\label{fig:SysErrEMCDetailed_EG2}
			 \end{figure}

			 \begin{figure}[t]
			  	\centering
			  	\includegraphics[width=0.49\textwidth]{figures/Meson/Systematics/EMC/SysMeanNewWithMean_Pi0_13TeVEG1.png}
			  	\includegraphics[width=0.49\textwidth]{figures/Meson/Systematics/EMC/SysMeanNewWithMean_Eta_13TeVEG1.png}
			  	\includegraphics[width=0.49\textwidth]{figures/Meson/Systematics/EMC/SysMeanNewWithMean_EtaToPi0_13TeVEG1.png}
			  	\caption{from the \ac{EMCal} method for \pp collisions at \s~=~13~TeV for the EG1 trigger. The colored points represent the individual error sources, while the black points represent the final systematic error for the $\pi^0$ meson transverse momentum spectrum.}
			  	\label{fig:SysErrEMCDetailed_EG1}
			  \end{figure}


% 		\begin{figure}[h]
% 			\centering
% 			\includegraphics[width=0.66\textwidth]{figures/Meson/Systematics/SysMeanNewWithMean_Pi0_8TeV_2017_11_02.eps}\\
% 			\includegraphics[width=0.66\textwidth]{figures/Meson/SystematicsTrigger/SysMeanNewWithMean_Pi0_8TeVEMC7_2017_11_02.eps}\\
% 			\includegraphics[width=0.66\textwidth]{figures/Meson/SystematicsTrigger/SysMeanNewWithMean_Pi0_8TeVEGA_2017_11_02.eps}
% 			\caption{Detailed systematic errors for \pp collisions at \sth. The colored points represent the individual error sources, while the black points represent the final systematic error for the $\pi^0$ meson transverse momentum spectrum.}
% 			\label{fig:SysErrPP2760Pi0Det}
% 		\end{figure}
%
% 		\begin{figure}
% 			\centering
% 			\includegraphics[width=0.7\textwidth]{figures/Meson/Systematics/SysMeanNewWithMean_Eta_8TeV_2017_11_02.eps}\\
% 			\includegraphics[width=0.7\textwidth]{figures/Meson/SystematicsTrigger/SysMeanNewWithMean_Eta_8TeVEMC7_2017_11_02.eps}\\
% 			\includegraphics[width=0.7\textwidth]{figures/Meson/SystematicsTrigger/SysMeanNewWithMean_Eta_8TeVEGA_2017_11_02.eps}
% 			\caption{Detailed systematic errors for \pp collisions at \sth. The colored points represent the individual error sources, while the black points represent the final systematic error for the $\eta$ meson transverse momentum spectrum.}
% 			\label{fig:SysErrPP2760EtaDet}
% 		\end{figure}
%
% 		\begin{figure}
% 			\centering
% 			\includegraphics[width=0.7\textwidth]{figures/Meson/Systematics/SysMeanNewWithMean_Pi0EtaBinning_8TeV_2017_11_02.eps}\\
% 			\includegraphics[width=0.7\textwidth]{figures/Meson/SystematicsTrigger/SysMeanNewWithMean_Pi0EtaBinning_8TeVEMC7_2017_11_02.eps}\\
% 			\includegraphics[width=0.7\textwidth]{figures/Meson/SystematicsTrigger/SysMeanNewWithMean_Pi0EtaBinning_8TeVEGA_2017_11_02.eps}
% 			\caption{Detailed systematic errors for \pp collisions at \sth. The colored points represent the individual error sources, while the black points represent the final systematic error for the \EtaToPi ratio versus transverse momentum.}
% 			\label{fig:SysErrPP2760EtaToPi0Det}
% 		\end{figure}
%
% 		\begin{figure}
% 			\centering
% 			\includegraphics[width=0.49\textwidth]{figures/Meson/Systematics/Pi0_data_Efficiencies_diffMCs.eps}
% 			\includegraphics[width=0.49\textwidth]{figures/Meson/Systematics/Pi0_data_CorrectedYield_MaterialError.eps}
% 			\caption{(left) Reconstruction efficiencies of the neutral pion for the two different generators used as well as the combination of both. (right) Comparison of the invariant yields when considering only \acs{EMCal} clusters with/without \acs{TRD} in front as well as considering the whole available \acs{EMCal} which is just the sum of the both specific cases.}
% 			\label{fig:sysEffMaterial}
% 		\end{figure}
%
% 		\begin{figure}
% 			\centering
% 			\includegraphics[width=0.49\textwidth]{figures/Meson/Systematics/Pi0_data_RAWYield_DistBC.eps}
% 			\includegraphics[width=0.49\textwidth]{figures/Meson/Systematics/Pi0_data_CorrectedYield_DistBC.eps}
% 			\caption{(left) Raw yields for neutral pion reconstruction for different distance to bad channel cuts. The different cuts applied are listed in the legend. (right) Comparison of the invariant yields for the different cuts. As it can be seen, they all agree well within uncertainties, although raw yields are lowered by 30~\% for certain cuts. Thus, a good description in Monte Carlo is proven so that no systematics may be assigned in this case.}
% 			\label{fig:sysDistBC}
% 		\end{figure}
%
% 		\begin{figure}
% 			\centering
% 		    \includegraphics[width=0.49\textwidth]{figures/Meson/Systematics/data_SPDPileUp.eps}
% 			\caption{The histogram is only filled if at least two primary vertices have been reconstructed; it shows the distance of the two primary vertices that have the highest distance in the event (unit of 'cm'). In blue, the rejection by SPD pileup cut is plotted whereas in green the rejection of SPD tracklet-cluster cut is shown. Assuming a Gaussian shape of the underlying distribution, the blue curve is fitted from -15 to 15 with a Gaussian (dotted black), excluding the center. Below 0.9~cm, the SPD pileup cut is not effective any more so that only a fraction of 'in-bunch' pileup can be rejected using the SPD tracklet-cluster cut, which is also fitted with a Gaussian (dotted red). In order to calculate the SPD pileup cut efficiency, the black dotted Gaussian is integrated from -0.9 to 0.9 and summed with the integral of red dotted Gaussian, integrated in the same region. This value is then divided by the black dotted Gaussian which is integrated from -15 to 15, arriving at the quoted efficiency of 92~\%.}
% 			\label{fig:SPDPileUp}
% 		\end{figure}

%		\begin{figure}
%			\centering
%			\includegraphics[width=0.49\textwidth]{figures/Meson/SystematicsTrigger/Pi0EtaBinning_data_CorrectedYield.eps}
%			\includegraphics[width=0.49\textwidth]{figures/Meson/SystematicsTrigger/Pi0_data_Mass.eps}
%			\caption{\newline Left plot: To properly consider the systematic error source due to the number of seven data taking periods, the neutral pion yield has been compared between the periods with enough statistics to do conclusive statements. The bins have been chosen to be wider than in the analysis in order to enhance the statistics per bin as we are dealing at single period level with approximately 10 times less statistics compared to the combined LHC12a-i analysis. Given the fluctuations around 1 and considering the statistical fluctuations, an error of 3\% has been concluded.\newline Right plot: The mass position for the same choice of data taking periods are shown here. Within statistical errors, all mass positions nicely agree to each other. It has to be noted that the statistics per single period is rather limited compared to the complete set of periods, so that especially at low \pT and higher \pT not all fits are converging due to limited number of neutral pion candidates per bin which explains the few given outliers in this plot.}
%			\label{fig:periodError}
%		\end{figure}
% \clearpage

\clearpage

\section{Acronyms and Technical Terms}
\begin{acronym}[ACORDEBB]
	\acro{ACORDE}{ALICE cosmic ray detector}
	\acro{AGS}{Alternating Gradient Synchrotron}
 	\acro{ALICE}{A Large Ion Collider Experiment}
	\acro{AOD}{Analysis Object Data}
	\acro{APD}{Avalanche Photodiode}
	\acro{ATLAS}{A Toroidal LHC Apparatus}
	\acro{BLUE}{Best Linear Unbiased Estimate}
	\acro{BNL}{Brockhaven National Laboratory}
	\acro{CERN}{European Organization for Nuclear Research}
	\acro{CDF}{Collider Detector at Fermilab}
	\acro{CMS}{Compact Muon Solenoid experiment}
	\acro{CCMF}{Conv-Calo mass fit}
	\acro{CMF}{Calo mass fit}
	\acro{CCRF}{Conv-Calo ratio fit}
	\acro{CRF}{Calo ratio fit}
	\acro{CPV}{Charged-Particle-Veto}
	\acro{CTP}{Central Trigger Processor}
	\acro{DAQ}{Data Acquisition}
	\acro{DCA}{distance of closest approach}
 	\acro{DCal}{Di-jet Calorimeter}
	\acro{DPM}{Dual Parton Model}
 	\acro{EMCal}{Electromagnetic Calorimeter}
	\acro{ESD}{Event Summary Data}
 	\acro{FEE}{Front End Electronics}
 	\acro{FMD}{Forward Multiplicity Detector}
 	\acro{FPGA}{Field Programmable Gate Array}
	\acro{FWHM}{full width at half maximum}
 	\acro{GEANT}{Geometry and Tracking Software}
	\acro{GTU}{Global Tracking Unit}
	\acro{HFE}{heavy flavor electron}
 	\acro{HG}{Hadron Gas}
 	\acro{HIJING}{Heavy Ion Jet Interaction Generator}
 	\acro{HLT}{High Level Trigger}
 	\acro{HMPID}{High Momentum Particle Identification Detector}
 	\acro{HYDJET}{Monte Carlo event generator for heavy ion collisions}
	\acro{INEL}{Inelastic}
 	\acro{ITS}{Inner Tracking System}
	\acro{L0}{level-0}
 	\acro{L1}{level-1}
 	\acro{L3}{High Energy Physics Experiment at LEP}
	\acro{LEP}{Large Electron Positron Collider}
	\acro{LEGO}{Lightweight Environment for Grid Operations}
	\acro{LEIR}{Low Energy Ion Ring}
	\acro{LHA}{Les Houches Accord}
	\acro{LHC}{Large Hadron Collider}
	\acro{LHCb}{LHC beauty experiment}
	\acro{LHEF}{Les Houches Event Files}
	\acro{LINAC2}{linear accelerator}
	\acro{LINAC3}{linear accelerator}
	\acro{LO}{Leading Order}
 	\acro{LTU}{Local Trigger Unit}
	\acro{LQCD}{Lattice QCD}
 	\acro{MC}{Monte Carlo simulation}
 	\acro{MRPC}{Multigap Resistive Plate Chamber}
	\acro{MWPC}{multi-wire-proportional chamber}
	\acro{NSD}{Non Single Diffractive Events}
	\acro{NLO}{Next-to-Leading Order}
	\acro{NNLO}{Next-to-Next-to Leading Order}
 	\acro{PDF}{Parton distribution function}
 	\acro{PHOS}{Photon Spectrometer}
 	\acro{RICH}{Ring Imaging Cherenkov detector}
	\acro{PCM}{Photon Conversion Method}
 	\acro{PDG}{Particle Data Group}
	\acro{PDF}{parton density functions}
	\acro{PHENIX}{Pioneering High Energy Nuclear Interactions eXperiment}
 	\acro{PID}{particle identification}
	\acro{PMD}{Photon Multiplicity Detector}
	\acro{pQCD}{perturbative QCD}
 	\acro{PS}{Proton Synchrotron}
	\acro{PSB}{Proton Synchrotron Booster}
	\acro{PYQUEN}{Monte Carlo event generator for pp-collisions}
	\acro{PYTHIA}{Monte Carlo event generator for pp-collisions}
	\acro{QCD}{Quantum Chromodynamics}
	\acro{QED}{Quantum Electrodynamics}
	\acro{QFT}{Quantum Field Theory}
	\acro{QGP}{Quark-Gluon Plasma}
	\acro{RCT}{run condition table}
	\acro{RHIC}{Relativistic Heavy Ion Collider}
	\acro{SDD}{Silicon Drift Detector}
	\acro{SDM}{Symmetric Decay Method}
	\acro{SHIM}{Secondary Hadronic Interaction Method}
	\acro{SM}{Standard Model}
	\acro{SPD}{Silicon Pixel Detector}
	\acro{SPS}{Super Proton Synchrotron}
	\acro{SSD}{Silicon Strip Detector}
	\acro{T0}[TZERO]{Timing and Trigger detector at ALICE}
	\acro{TBC}{Test beam corrected}
	\acro{TOF}{Time-Of-Flight detector}
	\acro{TPC}{Time Projection Chamber}
	\acro{TR}{transition radiation}
	\acro{TRD}{Transition Radiation Detector}
	\acro{TRG}{Trigger System}
	\acro{TRU}{Trigger Region Unit}
	\acro{V0}[V$^0$]{Unknown Particle}
	\acro{VZERO}{VZERO}
	\acro{ZDC}{Zero Degree Calorimeter}
	\acro{ZEM}{Zero Degree Electromagnetic Calorimeter}
	\acro{ZN}{Zero Degree Neutron Calorimeter}
	\acro{ZP}{Zero Degree Proton Calorimeter}
\end{acronym}

\clearpage

 \bibliography{../Bibliography.bib}
 \bibliographystyle{../kp.bst}


\end{appendix}

\end{document}
