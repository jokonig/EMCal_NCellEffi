\section{Additional Neutral Meson Plots}
\label{chap:AppNeutralMeson}

% 	\begin{figure}[h]
% 		\centering
% 			\includegraphics[width=0.5\textwidth]{figures/Meson/MesonTrigger/Pi0_data_CorrectedYieldFinal_WithMC.eps}\\
% 			\includegraphics[width=0.6\textwidth]{figures/Meson/MesonTrigger/Pi0_data_RatioSpectraToFitFinal_withMC.eps}
% 			\caption{Final measured neutral pion transverse momentum spectra with \ac{EMCal} for \pp collisions at \sth compared to the Monte Carlo input spectra (blue). Indeed, the input spectrum is a realistic one with a reasonable power law behavior, so that it is suitable for this analysis.}
% 			\label{fig:EMCalWithMC}
% 		\end{figure}

% \clearpage

\subsection{\texorpdfstring{Neutral Mesons in \pp Collisions at \s~=~13~TeV, MB INT7}{Neutral Mesons in pp Collisions at 13 TeV, MB INT7}}

% 		\begin{figure}[h]
% 		 	\includegraphics[width=0.96\textwidth]	{figures/Meson/INT7new/Pi0_data_MesonWithBck_00010113_00200009327000008250400000_1111111067032230000_0163103100000010.eps}\\
% 			\includegraphics[width=0.96\textwidth]{figures/Meson/INT7new/Pi0_data_MesonSubtracted_00010113_00200009327000008250400000_1111111067032230000_0163103100000010.eps}
% 			\caption{Invariant-mass distribution of reconstructed photon pairs $M_{\gamma\gamma}$ around the neutral pion mass in \pT slices in pp collisions at \sth  before (upper plot) and after subtraction (lower plot) for minimum bias INT7 triggered data. The black histograms in the upper plot show the combined signal and background distribution and the blue histograms show the calculated and normalized mixed event background. After the subtraction, the invariant mass distributions are fitted with \hyperref[eq:gausexp]{Equation~\ref*{eq:gausexp}}, which is shown in green in the lower plot.}
% 			\label{fig:InvMassPi0PtSliceINT7}
% 		\end{figure}
%
% 		\begin{figure}
% 		 	\includegraphics[width=0.96\textwidth]	{figures/Meson/INT7new/Eta_data_MesonWithBck_00010113_00200009327000008250400000_1111111067032230000_0163103100000010.eps}\\
% 			\includegraphics[width=0.96\textwidth]{figures/Meson/INT7new/Eta_data_MesonSubtracted_00010113_00200009327000008250400000_1111111067032230000_0163103100000010.eps}
% 			\caption{Invariant-mass distribution of reconstructed photon pairs $M_{\gamma\gamma}$ around the eta meson mass in \pT slices in pp collisions at \sth  before (upper plot) and after subtraction (lower plot) for minimum bias INT7 triggered data. The black histograms in the upper plot show the combined signal and background distribution and the blue histograms show the calculated and normalized mixed event background. After the subtraction, the invariant mass distributions are fitted with \hyperref[eq:gausexp]{Equation~\ref*{eq:gausexp}}, which is shown in green in the lower plot.}
% 			\label{fig:InvMassEtaPtSliceINT7}
% 		\end{figure}
%
% 		\begin{figure}
% 		 	\includegraphics[width=0.96\textwidth]	{figures/Meson/INT7new/Pi0_MesonLineShapeCompared_00010113_00200009327000008250400000_1111111067032230000_0163103100000010.eps}\\
% 			\includegraphics[width=0.96\textwidth]{figures/Meson/INT7new/Eta_MesonLineShapeCompared_00010113_00200009327000008250400000_1111111067032230000_0163103100000010.eps}
% 			\caption{Invariant-mass distribution of reconstructed photon pairs $M_{\gamma\gamma}$ around the neutral pion (upper) and eta meson (lower) mass in \pT slices in pp collisions at \sth after subtraction. The black dots represent the distribution obtained after the background subtraction in the data, while the blue dots show the same distribution for the reconstructed Monte Carlo sample. On top of these, the real neutral pion and eta distributions, according to the Monte Carlo, are plotted as red histograms, those were obtained by validating that both reconstructed photons are actually photons (or decay products of one) and originate in the same meson, which has to be a neutral pion or eta meson. All distributions are normalized the integral in the given invariant mass range.}
% 			\label{fig:InvMassShapeComp2760}
% 		\end{figure}
%
% 		\begin{figure}
% 		 	\includegraphics[width=0.96\textwidth]{figures/Meson/INT7new/Pi0_MC_TrueMesonDecomposedPhotonsAndElectron_00010113_00200009327000008250400000_1111111067032230000_0163103100000010.eps}
% 		 	\includegraphics[width=0.96\textwidth]	{figures/Meson/INT7new/Pi0_MC_TrueMesonDecomposedMerged_00010113_00200009327000008250400000_1111111067032230000_0163103100000010.eps}
% 			\caption{Invariant-mass distribution of reconstructed validated photon pairs $M_{\gamma\gamma}$ around the mass of the neutral pion mass (0.135~GeV/$c^2$) in \pT slices in pp collisions at \sth. The upper distributions have been obtained by using the Monte Carlo simulation to validate that the two photons come from the same neutral meson (black bullets). Afterwards it has been checked whether one (purple) or both (cyan) of the reconstructed photons in the calorimeter converted or if both clusters were from a real photon (red).
% 			For the lower plots, the same convention applies for the black points. However, this time is was checked whether the cluster reconstructed in the \ac{EMCal} contains more than one particle from the same neutral pion. In this case, the red bullets show the distribution where there are at least two particles from the same neutral pion in the same cluster, while the blue distribution in addition requires that at least one of them was originating in a conversion.}
% 			\label{fig:InvMassPi0PtSliceMCDec2760}
% 		\end{figure}
% 		\begin{figure}
% 			\includegraphics[width=0.96\textwidth]{figures/Meson/INT7new/Eta_MC_TrueMesonDecomposedPhotonsAndElectron_00010113_00200009327000008250400000_1111111067032230000_0163103100000010.eps}
% 			\includegraphics[width=0.96\textwidth]	{figures/Meson/INT7new/Eta_MC_TrueMesonDecomposedMerged_00010113_00200009327000008250400000_1111111067032230000_0163103100000010.eps}
% 			\caption{Invariant-mass distribution of reconstructed validated photon pairs $M_{\gamma\gamma}$ around the mass of the $\eta$ meson mass (0.548~GeV/$c^2$) in \pT slices in pp collisions at \sth. The upper distributions have been obtained by using the Monte Carlo simulation to validate that the two photons come from the same neutral meson (black bullets). Afterwards it has been checked whether one (purple) or both (cyan) of the reconstructed photons in the calorimeter converted or if both clusters were from a real photon (red). For the lower plots, the same convention applies for the black points. However, this time is was checked whether the cluster reconstructed in the \ac{EMCal} contains more than one particle from the same eta meson. In this case, the red bullets show the distribution where there are at least two particles from the same eta meson in the same cluster, while the blue distribution in addition requires that at least one of them was originating in a conversion.}
% 			\label{fig:InvMassEtaPtSliceMCDec2760}
% 		\end{figure}
%
% 			\begin{figure}
% 			\includegraphics[width=0.49\textwidth]{figures/Meson/INT7new/Weighting/Pi0_data_CorrectedYield.eps}
% 			\includegraphics[width=0.49\textwidth]{figures/Meson/INT7new/Weighting/Eta_data_CorrectedYield.eps}
% 				\caption{Corrected neutral meson spectra for the minimum bias INT7 trigger in \sth. For both mesons (pion on left side, eta on right side), the fully corrected spectra are compared using the multiplicity weighted correction factors and the ones without any modification.}
% 			\label{fig:CompMultWeight}
% 			\end{figure}
%
% 			\begin{figure}
% 			\centering
% 			\includegraphics[width=1\textwidth]{figures/Meson/INT7new/Timing/Pi0_data_MesonSubtracted_00000113_1111111003032220000_0163103100000050.eps}
% 			\caption{This figure shows the background subtracted invariant mass distributions for an \acs{EMCal}-\acs{EMCal} analysis where one cluster was taken from within the cluster time cut window and paired to another cluster that had to be outside the timing cut window. One can clearly see that there is no yield at all visible and therefore, all fits are failing. Thus, the cluster time cut is fully efficient and there is no yield found due to contributions from outside the timing cut.}
% 			\label{fig:EMCALInOutTiming}
%  			\end{figure}
% 		\clearpage


		\subsection{Detailed Systematics}
		\label{sec:SysDetailed}

			\begin{figure}[t]
				\centering
				\includegraphics[width=0.49\textwidth]{figures/Meson/Systematics/EMC/SysMeanNewWithMean_Pi0_13TeVINT7.png}
				\includegraphics[width=0.49\textwidth]{figures/Meson/Systematics/EMC/SysMeanNewWithMean_Eta_13TeVINT7.png}
				\includegraphics[width=0.49\textwidth]{figures/Meson/Systematics/EMC/SysMeanNewWithMean_EtaToPi0_13TeVINT7.png}
				\caption{from the \ac{EMCal} method for \pp collisions at \s~=~13~TeV for the INT7 trigger. The colored points represent the individual error sources, while the black points represent the final systematic error for the $\pi^0$ meson transverse momentum spectrum.}
				\label{fig:SysErrEMCDetailed_INT7}
			\end{figure}
			\begin{figure}[t]
				\centering
			  	\includegraphics[width=0.49\textwidth]{figures/Meson/Systematics/EMC/SysMeanNewWithMean_Pi0_13TeVEG2.png}
			  	\includegraphics[width=0.49\textwidth]{figures/Meson/Systematics/EMC/SysMeanNewWithMean_Eta_13TeVEG2.png}
			  	\includegraphics[width=0.49\textwidth]{figures/Meson/Systematics/EMC/SysMeanNewWithMean_EtaToPi0_13TeVEG2.png}
			  	\caption{from the \ac{EMCal} method for \pp collisions at \s~=~13~TeV for the EG2 trigger. The colored points represent the individual error sources, while the black points represent the final systematic error for the $\pi^0$ meson transverse momentum spectrum.}
			  	\label{fig:SysErrEMCDetailed_EG2}
			 \end{figure}

			 \begin{figure}[t]
			  	\centering
			  	\includegraphics[width=0.49\textwidth]{figures/Meson/Systematics/EMC/SysMeanNewWithMean_Pi0_13TeVEG1.png}
			  	\includegraphics[width=0.49\textwidth]{figures/Meson/Systematics/EMC/SysMeanNewWithMean_Eta_13TeVEG1.png}
			  	\includegraphics[width=0.49\textwidth]{figures/Meson/Systematics/EMC/SysMeanNewWithMean_EtaToPi0_13TeVEG1.png}
			  	\caption{from the \ac{EMCal} method for \pp collisions at \s~=~13~TeV for the EG1 trigger. The colored points represent the individual error sources, while the black points represent the final systematic error for the $\pi^0$ meson transverse momentum spectrum.}
			  	\label{fig:SysErrEMCDetailed_EG1}
			  \end{figure}


% 		\begin{figure}[h]
% 			\centering
% 			\includegraphics[width=0.66\textwidth]{figures/Meson/Systematics/SysMeanNewWithMean_Pi0_8TeV_2017_11_02.eps}\\
% 			\includegraphics[width=0.66\textwidth]{figures/Meson/SystematicsTrigger/SysMeanNewWithMean_Pi0_8TeVEMC7_2017_11_02.eps}\\
% 			\includegraphics[width=0.66\textwidth]{figures/Meson/SystematicsTrigger/SysMeanNewWithMean_Pi0_8TeVEGA_2017_11_02.eps}
% 			\caption{Detailed systematic errors for \pp collisions at \sth. The colored points represent the individual error sources, while the black points represent the final systematic error for the $\pi^0$ meson transverse momentum spectrum.}
% 			\label{fig:SysErrPP2760Pi0Det}
% 		\end{figure}
%
% 		\begin{figure}
% 			\centering
% 			\includegraphics[width=0.7\textwidth]{figures/Meson/Systematics/SysMeanNewWithMean_Eta_8TeV_2017_11_02.eps}\\
% 			\includegraphics[width=0.7\textwidth]{figures/Meson/SystematicsTrigger/SysMeanNewWithMean_Eta_8TeVEMC7_2017_11_02.eps}\\
% 			\includegraphics[width=0.7\textwidth]{figures/Meson/SystematicsTrigger/SysMeanNewWithMean_Eta_8TeVEGA_2017_11_02.eps}
% 			\caption{Detailed systematic errors for \pp collisions at \sth. The colored points represent the individual error sources, while the black points represent the final systematic error for the $\eta$ meson transverse momentum spectrum.}
% 			\label{fig:SysErrPP2760EtaDet}
% 		\end{figure}
%
% 		\begin{figure}
% 			\centering
% 			\includegraphics[width=0.7\textwidth]{figures/Meson/Systematics/SysMeanNewWithMean_Pi0EtaBinning_8TeV_2017_11_02.eps}\\
% 			\includegraphics[width=0.7\textwidth]{figures/Meson/SystematicsTrigger/SysMeanNewWithMean_Pi0EtaBinning_8TeVEMC7_2017_11_02.eps}\\
% 			\includegraphics[width=0.7\textwidth]{figures/Meson/SystematicsTrigger/SysMeanNewWithMean_Pi0EtaBinning_8TeVEGA_2017_11_02.eps}
% 			\caption{Detailed systematic errors for \pp collisions at \sth. The colored points represent the individual error sources, while the black points represent the final systematic error for the \EtaToPi ratio versus transverse momentum.}
% 			\label{fig:SysErrPP2760EtaToPi0Det}
% 		\end{figure}
%
% 		\begin{figure}
% 			\centering
% 			\includegraphics[width=0.49\textwidth]{figures/Meson/Systematics/Pi0_data_Efficiencies_diffMCs.eps}
% 			\includegraphics[width=0.49\textwidth]{figures/Meson/Systematics/Pi0_data_CorrectedYield_MaterialError.eps}
% 			\caption{(left) Reconstruction efficiencies of the neutral pion for the two different generators used as well as the combination of both. (right) Comparison of the invariant yields when considering only \acs{EMCal} clusters with/without \acs{TRD} in front as well as considering the whole available \acs{EMCal} which is just the sum of the both specific cases.}
% 			\label{fig:sysEffMaterial}
% 		\end{figure}
%
% 		\begin{figure}
% 			\centering
% 			\includegraphics[width=0.49\textwidth]{figures/Meson/Systematics/Pi0_data_RAWYield_DistBC.eps}
% 			\includegraphics[width=0.49\textwidth]{figures/Meson/Systematics/Pi0_data_CorrectedYield_DistBC.eps}
% 			\caption{(left) Raw yields for neutral pion reconstruction for different distance to bad channel cuts. The different cuts applied are listed in the legend. (right) Comparison of the invariant yields for the different cuts. As it can be seen, they all agree well within uncertainties, although raw yields are lowered by 30~\% for certain cuts. Thus, a good description in Monte Carlo is proven so that no systematics may be assigned in this case.}
% 			\label{fig:sysDistBC}
% 		\end{figure}
%
% 		\begin{figure}
% 			\centering
% 		    \includegraphics[width=0.49\textwidth]{figures/Meson/Systematics/data_SPDPileUp.eps}
% 			\caption{The histogram is only filled if at least two primary vertices have been reconstructed; it shows the distance of the two primary vertices that have the highest distance in the event (unit of 'cm'). In blue, the rejection by SPD pileup cut is plotted whereas in green the rejection of SPD tracklet-cluster cut is shown. Assuming a Gaussian shape of the underlying distribution, the blue curve is fitted from -15 to 15 with a Gaussian (dotted black), excluding the center. Below 0.9~cm, the SPD pileup cut is not effective any more so that only a fraction of 'in-bunch' pileup can be rejected using the SPD tracklet-cluster cut, which is also fitted with a Gaussian (dotted red). In order to calculate the SPD pileup cut efficiency, the black dotted Gaussian is integrated from -0.9 to 0.9 and summed with the integral of red dotted Gaussian, integrated in the same region. This value is then divided by the black dotted Gaussian which is integrated from -15 to 15, arriving at the quoted efficiency of 92~\%.}
% 			\label{fig:SPDPileUp}
% 		\end{figure}

%		\begin{figure}
%			\centering
%			\includegraphics[width=0.49\textwidth]{figures/Meson/SystematicsTrigger/Pi0EtaBinning_data_CorrectedYield.eps}
%			\includegraphics[width=0.49\textwidth]{figures/Meson/SystematicsTrigger/Pi0_data_Mass.eps}
%			\caption{\newline Left plot: To properly consider the systematic error source due to the number of seven data taking periods, the neutral pion yield has been compared between the periods with enough statistics to do conclusive statements. The bins have been chosen to be wider than in the analysis in order to enhance the statistics per bin as we are dealing at single period level with approximately 10 times less statistics compared to the combined LHC12a-i analysis. Given the fluctuations around 1 and considering the statistical fluctuations, an error of 3\% has been concluded.\newline Right plot: The mass position for the same choice of data taking periods are shown here. Within statistical errors, all mass positions nicely agree to each other. It has to be noted that the statistics per single period is rather limited compared to the complete set of periods, so that especially at low \pT and higher \pT not all fits are converging due to limited number of neutral pion candidates per bin which explains the few given outliers in this plot.}
%			\label{fig:periodError}
%		\end{figure}
% \clearpage
