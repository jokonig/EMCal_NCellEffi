\chapter{Introduction}
 	\begin{flushright}
		\Large
		\textit{"A scientist in his laboratory is not a mere technician:\\
					 he is also a child confronting natural phenomena that impress him\\
					 as though they were fairy tales.``\\
 	 	}\vspace{0.3cm}
		\footnotesize{ Marie Curie (1867 - 1934) }
 	\end{flushright}
\vspace{2cm}
As a child we see the world with different eyes, everything is fascinating and new. Every day new phenomena can be studied and although they might be well known to the adults, for a child they are exciting and it tries to explain them. While growing older most of these children stop asking questions and accept the world around them as it is and as it is told to be. Some of them, however, remain curious and some of these even become scientists, trying to unravel the mysteries and fairy tales of nature. Today our understanding of the world is rather good, we know the main mechanisms how the earth and universe developed over the last 14 billion years. Nevertheless, there are still white spots remaining and scientists from the different disciplines try to fill these. \\
The Large Hadron Collider (LHC), currently the most powerful particle accelerator ever built, was designed to study two of the most important questions for particle physicists: \textit{
\begin{itemize}
		\item  How did the universe evolve after its creation? 
		\item  What are the smallest building parts of nature?
\end{itemize}
}
Although these questions are located at opposite ends of the observation horizon, both can be studied at the LHC. In the accelerator protons can be collided with a center-of-mass energy of up to \s~=~14~TeV and heavy ions up to a \sNN~=~5.5~TeV per nucleon-nucleon pair. Such conditions were never reached before in a laboratory and, therefore, offer a new possibility to test current theories. Highly energetic \pp collisions not only allow to search for the Higgs boson but also to study the known particles to a new level of precision. In heavy-ion collisions, on the other hand, a state of strongly coupled matter is created, the quark-gluon plasma, which only existed shortly after the creation of the universe. Therefore, the evolution of matter shortly after the Big Bang can be investigated by several of the LHC experiments. \\
One of these experiments is the ALICE (A Large Ion Collider Experiment) detector system, in which context this thesis has been carried out. It has been designed to handle large charged-particle densities, while at the same time being able to identify these particle down to very low momenta. Thus it allows to study a variety of aspects of the produced strongly coupled medium. One of these aspects, the suppression of high-momentum particles due to the interactions in the medium will be studied in this thesis. It is quantified using the nuclear suppression factor $R_{AA}$, which is the ratio of the yield measured in \Pb collisions scaled by the number of binary collisions to the yield measured in \pp collisions. \\
This thesis describes the measurement of the neutral pion and eta meson production in \pp at \s~=~0.9,~2.76 and 7 TeV and \Pb collisions at \sNN = 2.76 TeV. Both mesons are detected in their two-photon decay channel and the photons are reconstructed using the photon conversion method. While the measured invariant cross sections in \pp collisions allow insights to the neutral pion and eta particle production, the measurement in \Pb collisions allows to distinguish between different energy loss scenarios in the created medium.\\

The structure of the thesis is as follows: After this short introduction, a theoretical overview together with the current state of knowledge from the experimental point of view will be presented (\hyperref[chap:theory]{Chapter~\ref*{chap:theory}}). Then an overview of the experimental setup (\hyperref[chap:experiment]{Chapter~\ref*{chap:experiment}}) is given and the photon reconstruction via the photon conversion method is explained (\hyperref[chap:PCM]{Chapter~\ref*{chap:PCM}}). \hyperref[chap:Material]{Chapter~\ref*{chap:Material}} is dedicated to the calculation of the error of the material budget in ALICE. The results on the neutral pion and eta meson measurements in \pp and \Pb collisions are presented in \hyperref[chap:Meson]{Chapter~\ref*{chap:Meson}}. The thesis concludes with a summary and outlook.
